\documentclass[ignorenonframetext,]{beamer}
\setbeamertemplate{caption}[numbered]
\setbeamertemplate{caption label separator}{: }
\setbeamercolor{caption name}{fg=normal text.fg}
\beamertemplatenavigationsymbolsempty
\usepackage{lmodern}
\usepackage{amssymb,amsmath}
\usepackage{ifxetex,ifluatex}
\usepackage{fixltx2e} % provides \textsubscript
\ifnum 0\ifxetex 1\fi\ifluatex 1\fi=0 % if pdftex
  \usepackage[T1]{fontenc}
  \usepackage[utf8]{inputenc}
\else % if luatex or xelatex
  \ifxetex
    \usepackage{mathspec}
  \else
    \usepackage{fontspec}
  \fi
  \defaultfontfeatures{Ligatures=TeX,Scale=MatchLowercase}
\fi
% use upquote if available, for straight quotes in verbatim environments
\IfFileExists{upquote.sty}{\usepackage{upquote}}{}
% use microtype if available
\IfFileExists{microtype.sty}{%
\usepackage{microtype}
\UseMicrotypeSet[protrusion]{basicmath} % disable protrusion for tt fonts
}{}
\newif\ifbibliography
\hypersetup{
            pdftitle={Intro to Git and GitHub},
            pdfauthor={Tara Henechowicz},
            pdfborder={0 0 0},
            breaklinks=true}
\urlstyle{same}  % don't use monospace font for urls

% Prevent slide breaks in the middle of a paragraph:
\widowpenalties 1 10000
\raggedbottom

\AtBeginPart{
  \let\insertpartnumber\relax
  \let\partname\relax
  \frame{\partpage}
}
\AtBeginSection{
  \ifbibliography
  \else
    \let\insertsectionnumber\relax
    \let\sectionname\relax
    \frame{\sectionpage}
  \fi
}
\AtBeginSubsection{
  \let\insertsubsectionnumber\relax
  \let\subsectionname\relax
  \frame{\subsectionpage}
}

\setlength{\parindent}{0pt}
\setlength{\parskip}{6pt plus 2pt minus 1pt}
\setlength{\emergencystretch}{3em}  % prevent overfull lines
\providecommand{\tightlist}{%
  \setlength{\itemsep}{0pt}\setlength{\parskip}{0pt}}
\setcounter{secnumdepth}{0}

\title{Intro to Git and GitHub}
\author{Tara Henechowicz}
\date{April 25, 2022}

\begin{document}
\frame{\titlepage}

\begin{frame}{Session Outline}

\begin{enumerate}
\def\labelenumi{\arabic{enumi}.}
\tightlist
\item
  Download and Install Git help, creating github account help (first 15
  minutes)
\item
  Introductions (15 minutes)
\item
  Git Slides (see slides from U of T coders) (20 minutes)
\item
  Working with Git locally: edit a text file and git it
\item
  Pushing to GitHub
\item
  GitHub features for Collaboration
\end{enumerate}

\end{frame}

\begin{frame}{Step 1. Configure git}

\end{frame}

\begin{frame}[fragile]{Step 2. Create a Repository on your computer with
gitGit}

\begin{verbatim}
## /Users/tarahenechowicz/Documents/RTeaching/IntroGitGitHub
## bash: line 7: cd: Documents: No such file or directory
## mkdir: GitPractice2: File exists
\end{verbatim}

\end{frame}

\begin{frame}[fragile]{Step 3. Git Init (Initialize the Repo)}

\begin{verbatim}
## Reinitialized existing Git repository in /Users/tarahenechowicz/Documents/RTeaching/IntroGitGitHub/.git/
## master
\end{verbatim}

\end{frame}

\begin{frame}{Adding files to the Repo}

\begin{itemize}
\tightlist
\item
  Let's create a text file to store some data to work on
\end{itemize}

\begin{enumerate}
\def\labelenumi{\arabic{enumi}.}
\item
  Create a text file named mydata.txt
\item
  Open the file and add:
\end{enumerate}

\begin{itemize}
\tightlist
\item
  your name
\item
  your program
\item
  your research area
\end{itemize}

\begin{enumerate}
\def\labelenumi{\arabic{enumi}.}
\setcounter{enumi}{2}
\tightlist
\item
  Save the file.
\end{enumerate}

\end{frame}

\begin{frame}[fragile]{Making our first commit}

\begin{verbatim}
## On branch master
## Changes not staged for commit:
##   (use "git add <file>..." to update what will be committed)
##   (use "git restore <file>..." to discard changes in working directory)
##  modified:   mydata.txt
## 
## no changes added to commit (use "git add" and/or "git commit -a")
\end{verbatim}

you can also use \texttt{git\ add\ -A} to add all the files in your
repository (remember we only have one right now)

\end{frame}

\begin{frame}[fragile]{Making our first commit}

Let's save our first set of changes using a commit. You need to put a
message with a commit.

\begin{verbatim}
## [master f6fdefa] Commit for the first time
##  1 file changed, 1 insertion(+), 1 deletion(-)
\end{verbatim}

\end{frame}

\begin{frame}{Making our second commit}

Now we are going to back to the text file and make a change. - for
example, I'm going to add my affiliation in the first line - you can add
your programming languages and experiences - or add a sentence about why
you want to use git

\end{frame}

\begin{frame}[fragile]{Check activity and changes with git status or git
diff}

\begin{verbatim}
## On branch master
## Changes not staged for commit:
##   (use "git add <file>..." to update what will be committed)
##   (use "git restore <file>..." to discard changes in working directory)
##  modified:   SessionPlan.Rmd
##  modified:   SessionPlan.log
##  modified:   SessionPlan.tex
## 
## no changes added to commit (use "git add" and/or "git commit -a")
## diff --git a/SessionPlan.Rmd b/SessionPlan.Rmd
## index 96e3edc..e212f50 100644
## --- a/SessionPlan.Rmd
## +++ b/SessionPlan.Rmd
## @@ -194,7 +194,7 @@ For this lesson we are going to clone your profile repo onto your device.
##  - This is an introduction to a complicated git topic - the more branches, the trickier it gets. 
##  - Use with caution with starting, better to focus on just using the main branch
##  
## -![](https://www.nobledesktop.com/image/gitresources/git-branches-merge.png)
## +
##  
##  #Make a new branch
##  
## diff --git a/SessionPlan.log b/SessionPlan.log
## index 8fb5dd1..69c71d0 100644
## --- a/SessionPlan.log
## +++ b/SessionPlan.log
## @@ -1,4 +1,4 @@
## -This is pdfTeX, Version 3.14159265-2.6-1.40.21 (TeX Live 2020) (preloaded format=pdflatex 2020.4.7)  25 APR 2022 10:46
## +This is pdfTeX, Version 3.14159265-2.6-1.40.21 (TeX Live 2020) (preloaded format=pdflatex 2020.4.7)  25 APR 2022 10:47
##  entering extended mode
##   restricted \write18 enabled.
##   %&-line parsing enabled.
## @@ -6,180 +6,344 @@ entering extended mode
##  (./SessionPlan.tex
##  LaTeX2e <2020-02-02> patch level 5
##  L3 programming layer <2020-03-06> (/usr/local/texlive/2020/texmf-dist/tex/latex
## -/base/article.cls
## -Document Class: article 2019/12/20 v1.4l Standard LaTeX document class
## -(/usr/local/texlive/2020/texmf-dist/tex/latex/base/size10.clo
## -File: size10.clo 2019/12/20 v1.4l Standard LaTeX file (size option)
## -)
## -\c@part=\count167
## -\c@section=\count168
## -\c@subsection=\count169
## -\c@subsubsection=\count170
## -\c@paragraph=\count171
## -\c@subparagraph=\count172
## -\c@figure=\count173
## -\c@table=\count174
## -\abovecaptionskip=\skip47
## -\belowcaptionskip=\skip48
## -\bibindent=\dimen134
## -) (/usr/local/texlive/2020/texmf-dist/tex/latex/lm/lmodern.sty
## -Package: lmodern 2009/10/30 v1.6 Latin Modern Fonts
## -LaTeX Font Info:    Overwriting symbol font `operators' in version `normal'
## -(Font)                  OT1/cmr/m/n --> OT1/lmr/m/n on input line 22.
## -LaTeX Font Info:    Overwriting symbol font `letters' in version `normal'
## -(Font)                  OML/cmm/m/it --> OML/lmm/m/it on input line 23.
## -LaTeX Font Info:    Overwriting symbol font `symbols' in version `normal'
## -(Font)                  OMS/cmsy/m/n --> OMS/lmsy/m/n on input line 24.
## -LaTeX Font Info:    Overwriting symbol font `largesymbols' in version `normal'
## -(Font)                  OMX/cmex/m/n --> OMX/lmex/m/n on input line 25.
## -LaTeX Font Info:    Overwriting symbol font `operators' in version `bold'
## -(Font)                  OT1/cmr/bx/n --> OT1/lmr/bx/n on input line 26.
## -LaTeX Font Info:    Overwriting symbol font `letters' in version `bold'
## -(Font)                  OML/cmm/b/it --> OML/lmm/b/it on input line 27.
## -LaTeX Font Info:    Overwriting symbol font `symbols' in version `bold'
## -(Font)                  OMS/cmsy/b/n --> OMS/lmsy/b/n on input line 28.
## -LaTeX Font Info:    Overwriting symbol font `largesymbols' in version `bold'
## -(Font)                  OMX/cmex/m/n --> OMX/lmex/m/n on input line 29.
## -LaTeX Font Info:    Overwriting math alphabet `\mathbf' in version `normal'
## -(Font)                  OT1/cmr/bx/n --> OT1/lmr/bx/n on input line 31.
## -LaTeX Font Info:    Overwriting math alphabet `\mathsf' in version `normal'
## -(Font)                  OT1/cmss/m/n --> OT1/lmss/m/n on input line 32.
## -LaTeX Font Info:    Overwriting math alphabet `\mathit' in version `normal'
## -(Font)                  OT1/cmr/m/it --> OT1/lmr/m/it on input line 33.
## -LaTeX Font Info:    Overwriting math alphabet `\mathtt' in version `normal'
## -(Font)                  OT1/cmtt/m/n --> OT1/lmtt/m/n on input line 34.
## -LaTeX Font Info:    Overwriting math alphabet `\mathbf' in version `bold'
## -(Font)                  OT1/cmr/bx/n --> OT1/lmr/bx/n on input line 35.
## -LaTeX Font Info:    Overwriting math alphabet `\mathsf' in version `bold'
## -(Font)                  OT1/cmss/bx/n --> OT1/lmss/bx/n on input line 36.
## -LaTeX Font Info:    Overwriting math alphabet `\mathit' in version `bold'
## -(Font)                  OT1/cmr/bx/it --> OT1/lmr/bx/it on input line 37.
## -LaTeX Font Info:    Overwriting math alphabet `\mathtt' in version `bold'
## -(Font)                  OT1/cmtt/m/n --> OT1/lmtt/m/n on input line 38.
## -) (/usr/local/texlive/2020/texmf-dist/tex/latex/amsfonts/amssymb.sty
## -Package: amssymb 2013/01/14 v3.01 AMS font symbols
## -(/usr/local/texlive/2020/texmf-dist/tex/latex/amsfonts/amsfonts.sty
## -Package: amsfonts 2013/01/14 v3.01 Basic AMSFonts support
## -\@emptytoks=\toks15
## -\symAMSa=\mathgroup4
## -\symAMSb=\mathgroup5
## -LaTeX Font Info:    Redeclaring math symbol \hbar on input line 98.
## -LaTeX Font Info:    Overwriting math alphabet `\mathfrak' in version `bold'
## -(Font)                  U/euf/m/n --> U/euf/b/n on input line 106.
## -)) (/usr/local/texlive/2020/texmf-dist/tex/latex/amsmath/amsmath.sty
## -Package: amsmath 2020/01/20 v2.17e AMS math features
## -\@mathmargin=\skip49
## -For additional information on amsmath, use the `?' option.
## -(/usr/local/texlive/2020/texmf-dist/tex/latex/amsmath/amstext.sty
## -Package: amstext 2000/06/29 v2.01 AMS text
## -(/usr/local/texlive/2020/texmf-dist/tex/latex/amsmath/amsgen.sty
## -File: amsgen.sty 1999/11/30 v2.0 generic functions
## -\@emptytoks=\toks16
## -\ex@=\dimen135
## -)) (/usr/local/texlive/2020/texmf-dist/tex/latex/amsmath/amsbsy.sty
## -Package: amsbsy 1999/11/29 v1.2d Bold Symbols
## -\pmbraise@=\dimen136
## -) (/usr/local/texlive/2020/texmf-dist/tex/latex/amsmath/amsopn.sty
## -Package: amsopn 2016/03/08 v2.02 operator names
## +/beamer/beamer.cls
## +Document Class: beamer 2019/09/29 v3.57 A class for typesetting presentations
## +(/usr/local/texlive/2020/texmf-dist/tex/latex/beamer/beamerbasemodes.sty (/usr/
## +local/texlive/2020/texmf-dist/tex/latex/etoolbox/etoolbox.sty
## +Package: etoolbox 2019/09/21 v2.5h e-TeX tools for LaTeX (JAW)
## +\etb@tempcnta=\count167
##  )
## -\inf@bad=\count175
## -LaTeX Info: Redefining \frac on input line 227.
## -\uproot@=\count176
## -\leftroot@=\count177
## -LaTeX Info: Redefining \overline on input line 389.
## -\classnum@=\count178
## -\DOTSCASE@=\count179
## -LaTeX Info: Redefining \ldots on input line 486.
## -LaTeX Info: Redefining \dots on input line 489.
## -LaTeX Info: Redefining \cdots on input line 610.
## -\Mathstrutbox@=\box45
## -\strutbox@=\box46
## -\big@size=\dimen137
## -LaTeX Font Info:    Redeclaring font encoding OML on input line 733.
## -LaTeX Font Info:    Redeclaring font encoding OMS on input line 734.
## -\macc@depth=\count180
## -\c@MaxMatrixCols=\count181
## -\dotsspace@=\muskip16
## -\c@parentequation=\count182
## -\dspbrk@lvl=\count183
## -\tag@help=\toks17
## -\row@=\count184
## -\column@=\count185
## -\maxfields@=\count186
## -\andhelp@=\toks18
## -\eqnshift@=\dimen138
## -\alignsep@=\dimen139
## -\tagshift@=\dimen140
## -\tagwidth@=\dimen141
## -\totwidth@=\dimen142
## -\lineht@=\dimen143
## -\@envbody=\toks19
## -\multlinegap=\skip50
## -\multlinetaggap=\skip51
## -\mathdisplay@stack=\toks20
## -LaTeX Info: Redefining \[ on input line 2859.
## -LaTeX Info: Redefining \] on input line 2860.
## -) (/usr/local/texlive/2020/texmf-dist/tex/generic/iftex/ifxetex.sty
## -Package: ifxetex 2019/10/25 v0.7 ifxetex legacy package. Use iftex instead.
## +\beamer@tempbox=\box45
## +\beamer@tempcount=\count168
## +\c@beamerpauses=\count169
## +(/usr/local/texlive/2020/texmf-dist/tex/latex/beamer/beamerbasedecode.sty
## +\beamer@slideinframe=\count170
## +\beamer@minimum=\count171
## +\beamer@decode@box=\box46
## +)
## +\beamer@commentbox=\box47
## +\beamer@modecount=\count172
## +) (/usr/local/texlive/2020/texmf-dist/tex/generic/iftex/ifpdf.sty
## +Package: ifpdf 2019/10/25 v3.4 ifpdf legacy package. Use iftex instead.
##  (/usr/local/texlive/2020/texmf-dist/tex/generic/iftex/iftex.sty
##  Package: iftex 2020/03/06 v1.0d TeX engine tests
## -)) (/usr/local/texlive/2020/texmf-dist/tex/generic/iftex/ifluatex.sty
## -Package: ifluatex 2019/10/25 v1.5 ifluatex legacy package. Use iftex instead.
## -) (/usr/local/texlive/2020/texmf-dist/tex/latex/base/fixltx2e.sty
## -Package: fixltx2e 2016/12/29 v2.1a fixes to LaTeX (obsolete)
## -Applying: [2015/01/01] Old fixltx2e package on input line 46.
## +))
## +\headdp=\dimen134
## +\footheight=\dimen135
## +\sidebarheight=\dimen136
## +\beamer@tempdim=\dimen137
## +\beamer@finalheight=\dimen138
## +\beamer@animht=\dimen139
## +\beamer@animdp=\dimen140
## +\beamer@animwd=\dimen141
## +\beamer@leftmargin=\dimen142
## +\beamer@rightmargin=\dimen143
## +\beamer@leftsidebar=\dimen144
## +\beamer@rightsidebar=\dimen145
## +\beamer@boxsize=\dimen146
## +\beamer@vboxoffset=\dimen147
## +\beamer@descdefault=\dimen148
## +\beamer@descriptionwidth=\dimen149
## +\beamer@lastskip=\skip47
## +\beamer@areabox=\box48
## +\beamer@animcurrent=\box49
## +\beamer@animshowbox=\box50
## +\beamer@sectionbox=\box51
## +\beamer@logobox=\box52
## +\beamer@linebox=\box53
## +\beamer@sectioncount=\count173
## +\beamer@subsubsectionmax=\count174
## +\beamer@subsectionmax=\count175
## +\beamer@sectionmax=\count176
## +\beamer@totalheads=\count177
## +\beamer@headcounter=\count178
## +\beamer@partstartpage=\count179
## +\beamer@sectionstartpage=\count180
## +\beamer@subsectionstartpage=\count181
## +\beamer@animationtempa=\count182
## +\beamer@animationtempb=\count183
## +\beamer@xpos=\count184
## +\beamer@ypos=\count185
## +\beamer@ypos@offset=\count186
## +\beamer@showpartnumber=\count187
## +\beamer@currentsubsection=\count188
## +\beamer@coveringdepth=\count189
## +\beamer@sectionadjust=\count190
## +\beamer@tocsectionnumber=\count191
## +(/usr/local/texlive/2020/texmf-dist/tex/latex/beamer/beamerbaseoptions.sty (/us
## +r/local/texlive/2020/texmf-dist/tex/latex/graphics/keyval.sty
## +Package: keyval 2014/10/28 v1.15 key=value parser (DPC)
## +\KV@toks@=\toks15
## +))
## +\beamer@paperwidth=\skip48
## +\beamer@paperheight=\skip49
## +(/usr/local/texlive/2020/texmf-dist/tex/latex/geometry/geometry.sty
## +Package: geometry 2020/01/02 v5.9 Page Geometry
## +(/usr/local/texlive/2020/texmf-dist/tex/generic/iftex/ifvtex.sty
## +Package: ifvtex 2019/10/25 v1.7 ifvtex legacy package. Use iftex instead.
## +)
## +\Gm@cnth=\count192
## +\Gm@cntv=\count193
## +\c@Gm@tempcnt=\count194
## +\Gm@bindingoffset=\dimen150
## +\Gm@wd@mp=\dimen151
## +\Gm@odd@mp=\dimen152
## +\Gm@even@mp=\dimen153
## +\Gm@layoutwidth=\dimen154
## +\Gm@layoutheight=\dimen155
## +\Gm@layouthoffset=\dimen156
## +\Gm@layoutvoffset=\dimen157
## +\Gm@dimlist=\toks16
## +) (/usr/local/texlive/2020/texmf-dist/tex/latex/base/size11.clo
## +File: size11.clo 2019/12/20 v1.4l Standard LaTeX file (size option)
## +) (/usr/local/texlive/2020/texmf-dist/tex/latex/pgf/basiclayer/pgfcore.sty (/us
## +r/local/texlive/2020/texmf-dist/tex/latex/graphics/graphicx.sty
## +Package: graphicx 2019/11/30 v1.2a Enhanced LaTeX Graphics (DPC,SPQR)
## +(/usr/local/texlive/2020/texmf-dist/tex/latex/graphics/graphics.sty
## +Package: graphics 2019/11/30 v1.4a Standard LaTeX Graphics (DPC,SPQR)
## +(/usr/local/texlive/2020/texmf-dist/tex/latex/graphics/trig.sty
## +Package: trig 2016/01/03 v1.10 sin cos tan (DPC)
## +) (/usr/local/texlive/2020/texmf-dist/tex/latex/graphics-cfg/graphics.cfg
## +File: graphics.cfg 2016/06/04 v1.11 sample graphics configuration
## +)
## +Package graphics Info: Driver file: pdftex.def on input line 105.
## +(/usr/local/texlive/2020/texmf-dist/tex/latex/graphics-def/pdftex.def
## +File: pdftex.def 2018/01/08 v1.0l Graphics/color driver for pdftex
## +))
## +\Gin@req@height=\dimen158
## +\Gin@req@width=\dimen159
## +) (/usr/local/texlive/2020/texmf-dist/tex/latex/pgf/systemlayer/pgfsys.sty (/us
## +r/local/texlive/2020/texmf-dist/tex/latex/pgf/utilities/pgfrcs.sty
## +(/usr/local/texlive/2020/texmf-dist/tex/generic/pgf/utilities/pgfutil-common.te
## +x
## +\pgfutil@everybye=\toks17
## +\pgfutil@tempdima=\dimen160
## +\pgfutil@tempdimb=\dimen161
##  
## -Package fixltx2e Warning: fixltx2e is not required with releases after 2015
## -(fixltx2e)                All fixes are now in the LaTeX kernel.
## -(fixltx2e)                See the latexrelease package for details.
## +(/usr/local/texlive/2020/texmf-dist/tex/generic/pgf/utilities/pgfutil-common-li
## +sts.tex))
## +(/usr/local/texlive/2020/texmf-dist/tex/generic/pgf/utilities/pgfutil-latex.def
## +\pgfutil@abb=\box54
## +(/usr/local/texlive/2020/texmf-dist/tex/latex/ms/everyshi.sty
## +Package: everyshi 2001/05/15 v3.00 EveryShipout Package (MS)
## +)) (/usr/local/texlive/2020/texmf-dist/tex/generic/pgf/utilities/pgfrcs.code.te
## +x (/usr/local/texlive/2020/texmf-dist/tex/generic/pgf/pgf.revision.tex)
## +Package: pgfrcs 2020/01/08 v3.1.5b (3.1.5b)
## +))
## +(/usr/local/texlive/2020/texmf-dist/tex/generic/pgf/systemlayer/pgfsys.code.tex
## +Package: pgfsys 2020/01/08 v3.1.5b (3.1.5b)
## +(/usr/local/texlive/2020/texmf-dist/tex/generic/pgf/utilities/pgfkeys.code.tex
## +\pgfkeys@pathtoks=\toks18
## +\pgfkeys@temptoks=\toks19
##  
## -Already applied: [0000/00/00] Old fixltx2e package on input line 53.
## -) (/usr/local/texlive/2020/texmf-dist/tex/latex/base/fontenc.sty
## -Package: fontenc 2020/02/11 v2.0o Standard LaTeX package
## -LaTeX Font Info:    Trying to load font information for T1+lmr on input line 11
## -2.
## -(/usr/local/texlive/2020/texmf-dist/tex/latex/lm/t1lmr.fd
## -File: t1lmr.fd 2009/10/30 v1.6 Font defs for Latin Modern
## -)) (/usr/local/texlive/2020/texmf-dist/tex/latex/base/inputenc.sty
## -Package: inputenc 2018/08/11 v1.3c Input encoding file
## -\inpenc@prehook=\toks21
## -\inpenc@posthook=\toks22
## -) (/usr/local/texlive/2020/texmf-dist/tex/latex/upquote/upquote.sty
## -Package: upquote 2012/04/19 v1.3 upright-quote and grave-accent glyphs in verba
## -tim
## -(/usr/local/texlive/2020/texmf-dist/tex/latex/base/textcomp.sty
## -Package: textcomp 2020/02/02 v2.0n Standard LaTeX package
## -)) (/usr/local/texlive/2020/texmf-dist/tex/latex/microtype/microtype.sty
## -Package: microtype 2019/11/18 v2.7d Micro-typographical refinements (RS)
## -(/usr/local/texlive/2020/texmf-dist/tex/latex/graphics/keyval.sty
## -Package: keyval 2014/10/28 v1.15 key=value parser (DPC)
## -\KV@toks@=\toks23
## +(/usr/local/texlive/2020/texmf-dist/tex/generic/pgf/utilities/pgfkeysfiltered.c
## +ode.tex
## +\pgfkeys@tmptoks=\toks20
## +))
## +\pgf@x=\dimen162
## +\pgf@y=\dimen163
## +\pgf@xa=\dimen164
## +\pgf@ya=\dimen165
## +\pgf@xb=\dimen166
## +\pgf@yb=\dimen167
## +\pgf@xc=\dimen168
## +\pgf@yc=\dimen169
## +\pgf@xd=\dimen170
## +\pgf@yd=\dimen171
## +\w@pgf@writea=\write3
## +\r@pgf@reada=\read2
## +\c@pgf@counta=\count195
## +\c@pgf@countb=\count196
## +\c@pgf@countc=\count197
## +\c@pgf@countd=\count198
## +\t@pgf@toka=\toks21
## +\t@pgf@tokb=\toks22
## +\t@pgf@tokc=\toks23
## +\pgf@sys@id@count=\count199
## +(/usr/local/texlive/2020/texmf-dist/tex/generic/pgf/systemlayer/pgf.cfg
## +File: pgf.cfg 2020/01/08 v3.1.5b (3.1.5b)
##  )
## -\MT@toks=\toks24
## -\MT@count=\count187
## -LaTeX Info: Redefining \textls on input line 790.
## -\MT@outer@kern=\dimen144
## -LaTeX Info: Redefining \textmicrotypecontext on input line 1354.
## -\MT@listname@count=\count188
## -(/usr/local/texlive/2020/texmf-dist/tex/latex/microtype/microtype-pdftex.def
## -File: microtype-pdftex.def 2019/11/18 v2.7d Definitions specific to pdftex (RS)
## +Driver file for pgf: pgfsys-pdftex.def
##  
## -LaTeX Info: Redefining \lsstyle on input line 914.
## -LaTeX Info: Redefining \lslig on input line 914.
## -\MT@outer@space=\skip52
## +(/usr/local/texlive/2020/texmf-dist/tex/generic/pgf/systemlayer/pgfsys-pdftex.d
## +ef
## +File: pgfsys-pdftex.def 2020/01/08 v3.1.5b (3.1.5b)
## +
## +(/usr/local/texlive/2020/texmf-dist/tex/generic/pgf/systemlayer/pgfsys-common-p
## +df.def
## +File: pgfsys-common-pdf.def 2020/01/08 v3.1.5b (3.1.5b)
## +)))
## +(/usr/local/texlive/2020/texmf-dist/tex/generic/pgf/systemlayer/pgfsyssoftpath.
## +code.tex
## +File: pgfsyssoftpath.code.tex 2020/01/08 v3.1.5b (3.1.5b)
## +\pgfsyssoftpath@smallbuffer@items=\count266
## +\pgfsyssoftpath@bigbuffer@items=\count267
##  )
## -Package microtype Info: Loading configuration file microtype.cfg.
## -(/usr/local/texlive/2020/texmf-dist/tex/latex/microtype/microtype.cfg
## -File: microtype.cfg 2019/11/18 v2.7d microtype main configuration file (RS)
## -)) (/usr/local/texlive/2020/texmf-dist/tex/latex/hyperref/hyperref.sty
## -Package: hyperref 2020/01/14 v7.00d Hypertext links for LaTeX
## -(/usr/local/texlive/2020/texmf-dist/tex/generic/ltxcmds/ltxcmds.sty
## -Package: ltxcmds 2019/12/15 v1.24 LaTeX kernel commands for general use (HO)
## -) (/usr/local/texlive/2020/texmf-dist/tex/latex/pdftexcmds/pdftexcmds.sty
## -Package: pdftexcmds 2019/11/24 v0.31 Utility functions of pdfTeX for LuaTeX (HO
## +(/usr/local/texlive/2020/texmf-dist/tex/generic/pgf/systemlayer/pgfsysprotocol.
## +code.tex
## +File: pgfsysprotocol.code.tex 2020/01/08 v3.1.5b (3.1.5b)
## +)) (/usr/local/texlive/2020/texmf-dist/tex/latex/xcolor/xcolor.sty
## +Package: xcolor 2016/05/11 v2.12 LaTeX color extensions (UK)
## +(/usr/local/texlive/2020/texmf-dist/tex/latex/graphics-cfg/color.cfg
## +File: color.cfg 2016/01/02 v1.6 sample color configuration
## +)
## +Package xcolor Info: Driver file: pdftex.def on input line 225.
## +Package xcolor Info: Model `cmy' substituted by `cmy0' on input line 1348.
## +Package xcolor Info: Model `hsb' substituted by `rgb' on input line 1352.
## +Package xcolor Info: Model `RGB' extended on input line 1364.
## +Package xcolor Info: Model `HTML' substituted by `rgb' on input line 1366.
## +Package xcolor Info: Model `Hsb' substituted by `hsb' on input line 1367.
## +Package xcolor Info: Model `tHsb' substituted by `hsb' on input line 1368.
## +Package xcolor Info: Model `HSB' substituted by `hsb' on input line 1369.
## +Package xcolor Info: Model `Gray' substituted by `gray' on input line 1370.
## +Package xcolor Info: Model `wave' substituted by `hsb' on input line 1371.
## +)
## +(/usr/local/texlive/2020/texmf-dist/tex/generic/pgf/basiclayer/pgfcore.code.tex
## +Package: pgfcore 2020/01/08 v3.1.5b (3.1.5b)
## +(/usr/local/texlive/2020/texmf-dist/tex/generic/pgf/math/pgfmath.code.tex (/usr
## +/local/texlive/2020/texmf-dist/tex/generic/pgf/math/pgfmathcalc.code.tex (/usr/
## +local/texlive/2020/texmf-dist/tex/generic/pgf/math/pgfmathutil.code.tex)
## +(/usr/local/texlive/2020/texmf-dist/tex/generic/pgf/math/pgfmathparser.code.tex
## +\pgfmath@dimen=\dimen172
## +\pgfmath@count=\count268
## +\pgfmath@box=\box55
## +\pgfmath@toks=\toks24
## +\pgfmath@stack@operand=\toks25
## +\pgfmath@stack@operation=\toks26
##  )
## +(/usr/local/texlive/2020/texmf-dist/tex/generic/pgf/math/pgfmathfunctions.code.
## +tex
## +(/usr/local/texlive/2020/texmf-dist/tex/generic/pgf/math/pgfmathfunctions.basic
## +.code.tex)
## +(/usr/local/texlive/2020/texmf-dist/tex/generic/pgf/math/pgfmathfunctions.trigo
## +nometric.code.tex)
## +(/usr/local/texlive/2020/texmf-dist/tex/generic/pgf/math/pgfmathfunctions.rando
## +m.code.tex)
## +(/usr/local/texlive/2020/texmf-dist/tex/generic/pgf/math/pgfmathfunctions.compa
## +rison.code.tex)
## +(/usr/local/texlive/2020/texmf-dist/tex/generic/pgf/math/pgfmathfunctions.base.
## +code.tex)
## +(/usr/local/texlive/2020/texmf-dist/tex/generic/pgf/math/pgfmathfunctions.round
## +.code.tex)
## +(/usr/local/texlive/2020/texmf-dist/tex/generic/pgf/math/pgfmathfunctions.misc.
## +code.tex)
## +(/usr/local/texlive/2020/texmf-dist/tex/generic/pgf/math/pgfmathfunctions.integ
## +erarithmetics.code.tex))) (/usr/local/texlive/2020/texmf-dist/tex/generic/pgf/m
## +ath/pgfmathfloat.code.tex
## +\c@pgfmathroundto@lastzeros=\count269
## +)) (/usr/local/texlive/2020/texmf-dist/tex/generic/pgf/math/pgfint.code.tex)
## +(/usr/local/texlive/2020/texmf-dist/tex/generic/pgf/basiclayer/pgfcorepoints.co
## +de.tex
## +File: pgfcorepoints.code.tex 2020/01/08 v3.1.5b (3.1.5b)
## +\pgf@picminx=\dimen173
## +\pgf@picmaxx=\dimen174
## +\pgf@picminy=\dimen175
## +\pgf@picmaxy=\dimen176
## +\pgf@pathminx=\dimen177
## +\pgf@pathmaxx=\dimen178
## +\pgf@pathminy=\dimen179
## +\pgf@pathmaxy=\dimen180
## +\pgf@xx=\dimen181
## +\pgf@xy=\dimen182
## +\pgf@yx=\dimen183
## +\pgf@yy=\dimen184
## +\pgf@zx=\dimen185
## +\pgf@zy=\dimen186
## +)
## +(/usr/local/texlive/2020/texmf-dist/tex/generic/pgf/basiclayer/pgfcorepathconst
## +ruct.code.tex
## +File: pgfcorepathconstruct.code.tex 2020/01/08 v3.1.5b (3.1.5b)
## +\pgf@path@lastx=\dimen187
## +\pgf@path@lasty=\dimen188
## +)
## +(/usr/local/texlive/2020/texmf-dist/tex/generic/pgf/basiclayer/pgfcorepathusage
## +.code.tex
## +File: pgfcorepathusage.code.tex 2020/01/08 v3.1.5b (3.1.5b)
## +\pgf@shorten@end@additional=\dimen189
## +\pgf@shorten@start@additional=\dimen190
## +)
## +(/usr/local/texlive/2020/texmf-dist/tex/generic/pgf/basiclayer/pgfcorescopes.co
## +de.tex
## +File: pgfcorescopes.code.tex 2020/01/08 v3.1.5b (3.1.5b)
## +\pgfpic=\box56
## +\pgf@hbox=\box57
## +\pgf@layerbox@main=\box58
## +\pgf@picture@serial@count=\count270
## +)
## +(/usr/local/texlive/2020/texmf-dist/tex/generic/pgf/basiclayer/pgfcoregraphicst
## +ate.code.tex
## +File: pgfcoregraphicstate.code.tex 2020/01/08 v3.1.5b (3.1.5b)
## +\pgflinewidth=\dimen191
## +)
## +(/usr/local/texlive/2020/texmf-dist/tex/generic/pgf/basiclayer/pgfcoretransform
## +ations.code.tex
## +File: pgfcoretransformations.code.tex 2020/01/08 v3.1.5b (3.1.5b)
## +\pgf@pt@x=\dimen192
## +\pgf@pt@y=\dimen193
## +\pgf@pt@temp=\dimen194
## +)
## +(/usr/local/texlive/2020/texmf-dist/tex/generic/pgf/basiclayer/pgfcorequick.cod
## +e.tex
## +File: pgfcorequick.code.tex 2020/01/08 v3.1.5b (3.1.5b)
## +)
## +(/usr/local/texlive/2020/texmf-dist/tex/generic/pgf/basiclayer/pgfcoreobjects.c
## +ode.tex
## +File: pgfcoreobjects.code.tex 2020/01/08 v3.1.5b (3.1.5b)
## +)
## +(/usr/local/texlive/2020/texmf-dist/tex/generic/pgf/basiclayer/pgfcorepathproce
## +ssing.code.tex
## +File: pgfcorepathprocessing.code.tex 2020/01/08 v3.1.5b (3.1.5b)
## +)
## +(/usr/local/texlive/2020/texmf-dist/tex/generic/pgf/basiclayer/pgfcorearrows.co
## +de.tex
## +File: pgfcorearrows.code.tex 2020/01/08 v3.1.5b (3.1.5b)
## +\pgfarrowsep=\dimen195
## +)
## +(/usr/local/texlive/2020/texmf-dist/tex/generic/pgf/basiclayer/pgfcoreshade.cod
## +e.tex
## +File: pgfcoreshade.code.tex 2020/01/08 v3.1.5b (3.1.5b)
## +\pgf@max=\dimen196
## +\pgf@sys@shading@range@num=\count271
## +\pgf@shadingcount=\count272
## +)
## +(/usr/local/texlive/2020/texmf-dist/tex/generic/pgf/basiclayer/pgfcoreimage.cod
## +e.tex
## +File: pgfcoreimage.code.tex 2020/01/08 v3.1.5b (3.1.5b)
## +
## +(/usr/local/texlive/2020/texmf-dist/tex/generic/pgf/basiclayer/pgfcoreexternal.
## +code.tex
## +File: pgfcoreexternal.code.tex 2020/01/08 v3.1.5b (3.1.5b)
## +\pgfexternal@startupbox=\box59
## +))
## +(/usr/local/texlive/2020/texmf-dist/tex/generic/pgf/basiclayer/pgfcorelayers.co
## +de.tex
## +File: pgfcorelayers.code.tex 2020/01/08 v3.1.5b (3.1.5b)
## +)
## +(/usr/local/texlive/2020/texmf-dist/tex/generic/pgf/basiclayer/pgfcoretranspare
## +ncy.code.tex
## +File: pgfcoretransparency.code.tex 2020/01/08 v3.1.5b (3.1.5b)
## +)
## +(/usr/local/texlive/2020/texmf-dist/tex/generic/pgf/basiclayer/pgfcorepatterns.
## +code.tex
## +File: pgfcorepatterns.code.tex 2020/01/08 v3.1.5b (3.1.5b)
## +)
## +(/usr/local/texlive/2020/texmf-dist/tex/generic/pgf/basiclayer/pgfcorerdf.code.
## +tex
## +File: pgfcorerdf.code.tex 2020/01/08 v3.1.5b (3.1.5b)
## +))) (/usr/local/texlive/2020/texmf-dist/tex/latex/pgf/utilities/xxcolor.sty
## +Package: xxcolor 2003/10/24 ver 0.1
## +\XC@nummixins=\count273
## +\XC@countmixins=\count274
## +) (/usr/local/texlive/2020/texmf-dist/tex/generic/atbegshi/atbegshi.sty
## +Package: atbegshi 2019/12/05 v1.19 At begin shipout hook (HO)
##  (/usr/local/texlive/2020/texmf-dist/tex/generic/infwarerr/infwarerr.sty
##  Package: infwarerr 2019/12/03 v1.5 Providing info/warning/error messages (HO)
## +) (/usr/local/texlive/2020/texmf-dist/tex/generic/ltxcmds/ltxcmds.sty
## +Package: ltxcmds 2019/12/15 v1.24 LaTeX kernel commands for general use (HO)
## +)) (/usr/local/texlive/2020/texmf-dist/tex/latex/hyperref/hyperref.sty
## +Package: hyperref 2020/01/14 v7.00d Hypertext links for LaTeX
## +(/usr/local/texlive/2020/texmf-dist/tex/latex/pdftexcmds/pdftexcmds.sty
## +Package: pdftexcmds 2019/11/24 v0.31 Utility functions of pdfTeX for LuaTeX (HO
##  )
##  Package pdftexcmds Info: \pdf@primitive is available.
##  Package pdftexcmds Info: \pdf@ifprimitive is available.
## @@ -199,9 +363,9 @@ Package: auxhook 2019-12-17 v1.6 Hooks for auxiliary files (HO)
##  ) (/usr/local/texlive/2020/texmf-dist/tex/latex/kvoptions/kvoptions.sty
##  Package: kvoptions 2019/11/29 v3.13 Key value format for package options (HO)
##  )
## -\@linkdim=\dimen145
## -\Hy@linkcounter=\count189
## -\Hy@pagecounter=\count190
## +\@linkdim=\dimen197
## +\Hy@linkcounter=\count275
## +\Hy@pagecounter=\count276
##  (/usr/local/texlive/2020/texmf-dist/tex/latex/hyperref/pd1enc.def
##  File: pd1enc.def 2020/01/14 v7.00d Hyperref: PDFDocEncoding definition (HO)
##  Now handling font encoding PD1 ...
## @@ -211,36 +375,33 @@ Package: intcalc 2019/12/15 v1.3 Expandable calculations with integers (HO)
##  ) (/usr/local/texlive/2020/texmf-dist/tex/generic/etexcmds/etexcmds.sty
##  Package: etexcmds 2019/12/15 v1.7 Avoid name clashes with e-TeX commands (HO)
##  )
## -\Hy@SavedSpaceFactor=\count191
## -Package hyperref Info: Option `unicode' set `true' on input line 4421.
## -(/usr/local/texlive/2020/texmf-dist/tex/latex/hyperref/puenc.def
## -File: puenc.def 2020/01/14 v7.00d Hyperref: PDF Unicode definition (HO)
## -Now handling font encoding PU ...
## -... no UTF-8 mapping file for font encoding PU
## -)
## +\Hy@SavedSpaceFactor=\count277
## +Package hyperref Info: Option `bookmarks' set `true' on input line 4421.
## +Package hyperref Info: Option `bookmarksopen' set `true' on input line 4421.
## +Package hyperref Info: Option `implicit' set `false' on input line 4421.
##  Package hyperref Info: Hyper figures OFF on input line 4547.
##  Package hyperref Info: Link nesting OFF on input line 4552.
##  Package hyperref Info: Hyper index ON on input line 4555.
##  Package hyperref Info: Plain pages OFF on input line 4562.
##  Package hyperref Info: Backreferencing OFF on input line 4567.
## -Package hyperref Info: Implicit mode ON; LaTeX internals redefined.
## +Package hyperref Info: Implicit mode OFF; no redefinition of LaTeX internals.
##  Package hyperref Info: Bookmarks ON on input line 4800.
## -\c@Hy@tempcnt=\count192
## +\c@Hy@tempcnt=\count278
##  (/usr/local/texlive/2020/texmf-dist/tex/latex/url/url.sty
## -\Urlmuskip=\muskip17
## +\Urlmuskip=\muskip16
##  Package: url 2013/09/16  ver 3.4  Verb mode for urls, etc.
##  )
##  LaTeX Info: Redefining \url on input line 5159.
## -\XeTeXLinkMargin=\dimen146
## +\XeTeXLinkMargin=\dimen198
##  (/usr/local/texlive/2020/texmf-dist/tex/generic/bitset/bitset.sty
##  Package: bitset 2019/12/09 v1.3 Handle bit-vector datatype (HO)
##  (/usr/local/texlive/2020/texmf-dist/tex/generic/bigintcalc/bigintcalc.sty
##  Package: bigintcalc 2019/12/15 v1.5 Expandable calculations on big integers (HO
##  )
##  ))
## -\Fld@menulength=\count193
## -\Field@Width=\dimen147
## -\Fld@charsize=\dimen148
## +\Fld@menulength=\count279
## +\Field@Width=\dimen199
## +\Fld@charsize=\dimen256
##  Package hyperref Info: Hyper figures OFF on input line 6430.
##  Package hyperref Info: Link nesting OFF on input line 6435.
##  Package hyperref Info: Hyper index ON on input line 6438.
## @@ -250,22 +411,19 @@ Package hyperref Info: Link coloring with OCG OFF on input line 6455.
##  Package hyperref Info: PDF/A mode OFF on input line 6460.
##  LaTeX Info: Redefining \ref on input line 6500.
##  LaTeX Info: Redefining \pageref on input line 6504.
## -(/usr/local/texlive/2020/texmf-dist/tex/generic/atbegshi/atbegshi.sty
## -Package: atbegshi 2019/12/05 v1.19 At begin shipout hook (HO)
## -)
## -\Hy@abspage=\count194
## -\c@Item=\count195
## -\c@Hfootnote=\count196
## +\Hy@abspage=\count280
## +
## +Package hyperref Message: Stopped early.
## +
##  )
##  Package hyperref Info: Driver (autodetected): hpdftex.
##  (/usr/local/texlive/2020/texmf-dist/tex/latex/hyperref/hpdftex.def
##  File: hpdftex.def 2020/01/14 v7.00d Hyperref driver for pdfTeX
##  (/usr/local/texlive/2020/texmf-dist/tex/latex/atveryend/atveryend.sty
##  Package: atveryend 2019-12-11 v1.11 Hooks at the very end of document (HO)
## -Package atveryend Info: \enddocument detected (standard20110627).
##  )
## -\Fld@listcount=\count197
## -\c@bookmark@seq@number=\count198
## +\Fld@listcount=\count281
## +\c@bookmark@seq@number=\count282
##  
##  (/usr/local/texlive/2020/texmf-dist/tex/latex/rerunfilecheck/rerunfilecheck.sty
##  Package: rerunfilecheck 2019/12/05 v1.9 Rerun checks for auxiliary files (HO)
## @@ -275,138 +433,314 @@ Package: uniquecounter 2019/12/15 v1.4 Provide unlimited unique counter (HO)
##  )
##  Package uniquecounter Info: New unique counter `rerunfilecheck' on input line 2
##  86.
## +)) (/usr/local/texlive/2020/texmf-dist/tex/latex/beamer/beamerbaserequires.sty
## +(/usr/local/texlive/2020/texmf-dist/tex/latex/beamer/beamerbasecompatibility.st
## +y) (/usr/local/texlive/2020/texmf-dist/tex/latex/beamer/beamerbasefont.sty (/us
## +r/local/texlive/2020/texmf-dist/tex/latex/amsfonts/amssymb.sty
## +Package: amssymb 2013/01/14 v3.01 AMS font symbols
## +(/usr/local/texlive/2020/texmf-dist/tex/latex/amsfonts/amsfonts.sty
## +Package: amsfonts 2013/01/14 v3.01 Basic AMSFonts support
## +\@emptytoks=\toks27
## +\symAMSa=\mathgroup4
## +\symAMSb=\mathgroup5
## +LaTeX Font Info:    Redeclaring math symbol \hbar on input line 98.
## +LaTeX Font Info:    Overwriting math alphabet `\mathfrak' in version `bold'
## +(Font)                  U/euf/m/n --> U/euf/b/n on input line 106.
## +))
## +(/usr/local/texlive/2020/texmf-dist/tex/latex/sansmathaccent/sansmathaccent.sty
## +Package: sansmathaccent 2020/01/31
## +(/usr/local/texlive/2020/texmf-dist/tex/latex/koma-script/scrlfile.sty
## +Package: scrlfile 2020/01/24 v3.29 KOMA-Script package (loading files)
## +))) (/usr/local/texlive/2020/texmf-dist/tex/latex/beamer/beamerbasetranslator.s
## +ty (/usr/local/texlive/2020/texmf-dist/tex/latex/translator/translator.sty
## +Package: translator 2019-05-31 v1.12a Easy translation of strings in LaTeX
## +)) (/usr/local/texlive/2020/texmf-dist/tex/latex/beamer/beamerbasemisc.sty) (/u
## +sr/local/texlive/2020/texmf-dist/tex/latex/beamer/beamerbasetwoscreens.sty) (/u
## +sr/local/texlive/2020/texmf-dist/tex/latex/beamer/beamerbaseoverlay.sty
## +\beamer@argscount=\count283
## +\beamer@lastskipcover=\skip50
## +\beamer@trivlistdepth=\count284
## +) (/usr/local/texlive/2020/texmf-dist/tex/latex/beamer/beamerbasetitle.sty) (/u
## +sr/local/texlive/2020/texmf-dist/tex/latex/beamer/beamerbasesection.sty
## +\c@lecture=\count285
## +\c@part=\count286
## +\c@section=\count287
## +\c@subsection=\count288
## +\c@subsubsection=\count289
## +) (/usr/local/texlive/2020/texmf-dist/tex/latex/beamer/beamerbaseframe.sty
## +\beamer@framebox=\box60
## +\beamer@frametitlebox=\box61
## +\beamer@zoombox=\box62
## +\beamer@zoomcount=\count290
## +\beamer@zoomframecount=\count291
## +\beamer@frametextheight=\dimen257
## +\c@subsectionslide=\count292
## +\beamer@frametopskip=\skip51
## +\beamer@framebottomskip=\skip52
## +\beamer@frametopskipautobreak=\skip53
## +\beamer@framebottomskipautobreak=\skip54
## +\beamer@envbody=\toks28
## +\framewidth=\dimen258
## +\c@framenumber=\count293
## +) (/usr/local/texlive/2020/texmf-dist/tex/latex/beamer/beamerbaseverbatim.sty
## +\beamer@verbatimfileout=\write4
## +) (/usr/local/texlive/2020/texmf-dist/tex/latex/beamer/beamerbaseframesize.sty
## +\beamer@splitbox=\box63
## +\beamer@autobreakcount=\count294
## +\beamer@autobreaklastheight=\dimen259
## +\beamer@frametitletoks=\toks29
## +\beamer@framesubtitletoks=\toks30
##  )
## -\Hy@SectionHShift=\skip53
## +(/usr/local/texlive/2020/texmf-dist/tex/latex/beamer/beamerbaseframecomponents.
## +sty
## +\beamer@footins=\box64
## +) (/usr/local/texlive/2020/texmf-dist/tex/latex/beamer/beamerbasecolor.sty) (/u
## +sr/local/texlive/2020/texmf-dist/tex/latex/beamer/beamerbasenotes.sty
## +\beamer@frameboxcopy=\box65
## +) (/usr/local/texlive/2020/texmf-dist/tex/latex/beamer/beamerbasetoc.sty) (/usr
## +/local/texlive/2020/texmf-dist/tex/latex/beamer/beamerbasetemplates.sty
## +\beamer@sbttoks=\toks31
## +
## +(/usr/local/texlive/2020/texmf-dist/tex/latex/beamer/beamerbaseauxtemplates.sty
## +(/usr/local/texlive/2020/texmf-dist/tex/latex/beamer/beamerbaseboxes.sty
## +\bmb@box=\box66
## +\bmb@colorbox=\box67
## +\bmb@boxshadow=\box68
## +\bmb@boxshadowball=\box69
## +\bmb@boxshadowballlarge=\box70
## +\bmb@temp=\dimen260
## +\bmb@dima=\dimen261
## +\bmb@dimb=\dimen262
## +\bmb@prevheight=\dimen263
##  )
## -Package hyperref Info: Option `breaklinks' set `true' on input line 30.
## -(/usr/local/texlive/2020/texmf-dist/tex/latex/geometry/geometry.sty
## -Package: geometry 2020/01/02 v5.9 Page Geometry
## -(/usr/local/texlive/2020/texmf-dist/tex/generic/iftex/ifvtex.sty
## -Package: ifvtex 2019/10/25 v1.7 ifvtex legacy package. Use iftex instead.
## +\beamer@blockheadheight=\dimen264
## +))
## +(/usr/local/texlive/2020/texmf-dist/tex/latex/beamer/beamerbaselocalstructure.s
## +ty (/usr/local/texlive/2020/texmf-dist/tex/latex/tools/enumerate.sty
## +Package: enumerate 2015/07/23 v3.00 enumerate extensions (DPC)
## +\@enLab=\toks32
##  )
## -\Gm@cnth=\count199
## -\Gm@cntv=\count266
## -\c@Gm@tempcnt=\count267
## -\Gm@bindingoffset=\dimen149
## -\Gm@wd@mp=\dimen150
## -\Gm@odd@mp=\dimen151
## -\Gm@even@mp=\dimen152
## -\Gm@layoutwidth=\dimen153
## -\Gm@layoutheight=\dimen154
## -\Gm@layouthoffset=\dimen155
## -\Gm@layoutvoffset=\dimen156
## -\Gm@dimlist=\toks25
## -) (/usr/local/texlive/2020/texmf-dist/tex/latex/graphics/graphicx.sty
## -Package: graphicx 2019/11/30 v1.2a Enhanced LaTeX Graphics (DPC,SPQR)
## -(/usr/local/texlive/2020/texmf-dist/tex/latex/graphics/graphics.sty
## -Package: graphics 2019/11/30 v1.4a Standard LaTeX Graphics (DPC,SPQR)
## -(/usr/local/texlive/2020/texmf-dist/tex/latex/graphics/trig.sty
## -Package: trig 2016/01/03 v1.10 sin cos tan (DPC)
## -) (/usr/local/texlive/2020/texmf-dist/tex/latex/graphics-cfg/graphics.cfg
## -File: graphics.cfg 2016/06/04 v1.11 sample graphics configuration
## +\c@figure=\count295
## +\c@table=\count296
## +\abovecaptionskip=\skip55
## +\belowcaptionskip=\skip56
## +) (/usr/local/texlive/2020/texmf-dist/tex/latex/beamer/beamerbasenavigation.sty
## +\beamer@section@min@dim=\dimen265
## +) (/usr/local/texlive/2020/texmf-dist/tex/latex/beamer/beamerbasetheorems.sty (
## +/usr/local/texlive/2020/texmf-dist/tex/latex/amsmath/amsmath.sty
## +Package: amsmath 2020/01/20 v2.17e AMS math features
## +\@mathmargin=\skip57
## +For additional information on amsmath, use the `?' option.
## +(/usr/local/texlive/2020/texmf-dist/tex/latex/amsmath/amstext.sty
## +Package: amstext 2000/06/29 v2.01 AMS text
## +(/usr/local/texlive/2020/texmf-dist/tex/latex/amsmath/amsgen.sty
## +File: amsgen.sty 1999/11/30 v2.0 generic functions
## +\@emptytoks=\toks33
## +\ex@=\dimen266
## +)) (/usr/local/texlive/2020/texmf-dist/tex/latex/amsmath/amsbsy.sty
## +Package: amsbsy 1999/11/29 v1.2d Bold Symbols
## +\pmbraise@=\dimen267
## +) (/usr/local/texlive/2020/texmf-dist/tex/latex/amsmath/amsopn.sty
## +Package: amsopn 2016/03/08 v2.02 operator names
##  )
## -Package graphics Info: Driver file: pdftex.def on input line 105.
## -(/usr/local/texlive/2020/texmf-dist/tex/latex/graphics-def/pdftex.def
## -File: pdftex.def 2018/01/08 v1.0l Graphics/color driver for pdftex
## +\inf@bad=\count297
## +LaTeX Info: Redefining \frac on input line 227.
## +\uproot@=\count298
## +\leftroot@=\count299
## +LaTeX Info: Redefining \overline on input line 389.
## +\classnum@=\count300
## +\DOTSCASE@=\count301
## +LaTeX Info: Redefining \ldots on input line 486.
## +LaTeX Info: Redefining \dots on input line 489.
## +LaTeX Info: Redefining \cdots on input line 610.
## +\Mathstrutbox@=\box71
## +\strutbox@=\box72
## +\big@size=\dimen268
## +LaTeX Font Info:    Redeclaring font encoding OML on input line 733.
## +LaTeX Font Info:    Redeclaring font encoding OMS on input line 734.
## +\macc@depth=\count302
## +\c@MaxMatrixCols=\count303
## +\dotsspace@=\muskip17
## +\c@parentequation=\count304
## +\dspbrk@lvl=\count305
## +\tag@help=\toks34
## +\row@=\count306
## +\column@=\count307
## +\maxfields@=\count308
## +\andhelp@=\toks35
## +\eqnshift@=\dimen269
## +\alignsep@=\dimen270
## +\tagshift@=\dimen271
## +\tagwidth@=\dimen272
## +\totwidth@=\dimen273
## +\lineht@=\dimen274
## +\@envbody=\toks36
## +\multlinegap=\skip58
## +\multlinetaggap=\skip59
## +\mathdisplay@stack=\toks37
## +LaTeX Info: Redefining \[ on input line 2859.
## +LaTeX Info: Redefining \] on input line 2860.
## +) (/usr/local/texlive/2020/texmf-dist/tex/latex/amscls/amsthm.sty
## +Package: amsthm 2017/10/31 v2.20.4
## +\thm@style=\toks38
## +\thm@bodyfont=\toks39
## +\thm@headfont=\toks40
## +\thm@notefont=\toks41
## +\thm@headpunct=\toks42
## +\thm@preskip=\skip60
## +\thm@postskip=\skip61
## +\thm@headsep=\skip62
## +\dth@everypar=\toks43
## +)
## +\c@theorem=\count309
## +) (/usr/local/texlive/2020/texmf-dist/tex/latex/beamer/beamerbasethemes.sty)) (
## +/usr/local/texlive/2020/texmf-dist/tex/latex/beamer/beamerthemedefault.sty
## +(/usr/local/texlive/2020/texmf-dist/tex/latex/beamer/beamerfontthemedefault.sty
## +)
## +(/usr/local/texlive/2020/texmf-dist/tex/latex/beamer/beamercolorthemedefault.st
## +y)
## +(/usr/local/texlive/2020/texmf-dist/tex/latex/beamer/beamerinnerthemedefault.st
## +y
## +\beamer@dima=\dimen275
## +\beamer@dimb=\dimen276
## +)
## +(/usr/local/texlive/2020/texmf-dist/tex/latex/beamer/beamerouterthemedefault.st
## +y))) (/usr/local/texlive/2020/texmf-dist/tex/latex/lm/lmodern.sty
## +Package: lmodern 2009/10/30 v1.6 Latin Modern Fonts
## +LaTeX Font Info:    Overwriting symbol font `operators' in version `normal'
## +(Font)                  OT1/cmr/m/n --> OT1/lmr/m/n on input line 22.
## +LaTeX Font Info:    Overwriting symbol font `letters' in version `normal'
## +(Font)                  OML/cmm/m/it --> OML/lmm/m/it on input line 23.
## +LaTeX Font Info:    Overwriting symbol font `symbols' in version `normal'
## +(Font)                  OMS/cmsy/m/n --> OMS/lmsy/m/n on input line 24.
## +LaTeX Font Info:    Overwriting symbol font `largesymbols' in version `normal'
## +(Font)                  OMX/cmex/m/n --> OMX/lmex/m/n on input line 25.
## +LaTeX Font Info:    Overwriting symbol font `operators' in version `bold'
## +(Font)                  OT1/cmr/bx/n --> OT1/lmr/bx/n on input line 26.
## +LaTeX Font Info:    Overwriting symbol font `letters' in version `bold'
## +(Font)                  OML/cmm/b/it --> OML/lmm/b/it on input line 27.
## +LaTeX Font Info:    Overwriting symbol font `symbols' in version `bold'
## +(Font)                  OMS/cmsy/b/n --> OMS/lmsy/b/n on input line 28.
## +LaTeX Font Info:    Overwriting symbol font `largesymbols' in version `bold'
## +(Font)                  OMX/cmex/m/n --> OMX/lmex/m/n on input line 29.
## +LaTeX Font Info:    Overwriting math alphabet `\mathbf' in version `normal'
## +(Font)                  OT1/cmr/bx/n --> OT1/lmr/bx/n on input line 31.
## +LaTeX Font Info:    Overwriting math alphabet `\mathsf' in version `normal'
## +(Font)                  OT1/cmss/m/n --> OT1/lmss/m/n on input line 32.
## +LaTeX Font Info:    Overwriting math alphabet `\mathit' in version `normal'
## +(Font)                  OT1/cmr/m/it --> OT1/lmr/m/it on input line 33.
## +LaTeX Font Info:    Overwriting math alphabet `\mathtt' in version `normal'
## +(Font)                  OT1/cmtt/m/n --> OT1/lmtt/m/n on input line 34.
## +LaTeX Font Info:    Overwriting math alphabet `\mathbf' in version `bold'
## +(Font)                  OT1/cmr/bx/n --> OT1/lmr/bx/n on input line 35.
## +LaTeX Font Info:    Overwriting math alphabet `\mathsf' in version `bold'
## +(Font)                  OT1/cmss/bx/n --> OT1/lmss/bx/n on input line 36.
## +LaTeX Font Info:    Overwriting math alphabet `\mathit' in version `bold'
## +(Font)                  OT1/cmr/bx/it --> OT1/lmr/bx/it on input line 37.
## +LaTeX Font Info:    Overwriting math alphabet `\mathtt' in version `bold'
## +(Font)                  OT1/cmtt/m/n --> OT1/lmtt/m/n on input line 38.
## +) (/usr/local/texlive/2020/texmf-dist/tex/generic/iftex/ifxetex.sty
## +Package: ifxetex 2019/10/25 v0.7 ifxetex legacy package. Use iftex instead.
## +) (/usr/local/texlive/2020/texmf-dist/tex/generic/iftex/ifluatex.sty
## +Package: ifluatex 2019/10/25 v1.5 ifluatex legacy package. Use iftex instead.
## +) (/usr/local/texlive/2020/texmf-dist/tex/latex/base/fixltx2e.sty
## +Package: fixltx2e 2016/12/29 v2.1a fixes to LaTeX (obsolete)
## +Applying: [2015/01/01] Old fixltx2e package on input line 46.
## +
## +Package fixltx2e Warning: fixltx2e is not required with releases after 2015
## +(fixltx2e)                All fixes are now in the LaTeX kernel.
## +(fixltx2e)                See the latexrelease package for details.
## +
## +Already applied: [0000/00/00] Old fixltx2e package on input line 53.
## +) (/usr/local/texlive/2020/texmf-dist/tex/latex/base/fontenc.sty
## +Package: fontenc 2020/02/11 v2.0o Standard LaTeX package
## +LaTeX Font Info:    Trying to load font information for T1+lmss on input line 1
## +12.
## +(/usr/local/texlive/2020/texmf-dist/tex/latex/lm/t1lmss.fd
## +File: t1lmss.fd 2009/10/30 v1.6 Font defs for Latin Modern
## +)) (/usr/local/texlive/2020/texmf-dist/tex/latex/base/inputenc.sty
## +Package: inputenc 2018/08/11 v1.3c Input encoding file
## +\inpenc@prehook=\toks44
## +\inpenc@posthook=\toks45
## +) (/usr/local/texlive/2020/texmf-dist/tex/latex/upquote/upquote.sty
## +Package: upquote 2012/04/19 v1.3 upright-quote and grave-accent glyphs in verba
## +tim
## +(/usr/local/texlive/2020/texmf-dist/tex/latex/base/textcomp.sty
## +Package: textcomp 2020/02/02 v2.0n Standard LaTeX package
## +)) (/usr/local/texlive/2020/texmf-dist/tex/latex/microtype/microtype.sty
## +Package: microtype 2019/11/18 v2.7d Micro-typographical refinements (RS)
## +\MT@toks=\toks46
## +\MT@count=\count310
## +LaTeX Info: Redefining \textls on input line 790.
## +\MT@outer@kern=\dimen277
## +LaTeX Info: Redefining \textmicrotypecontext on input line 1354.
## +\MT@listname@count=\count311
## +(/usr/local/texlive/2020/texmf-dist/tex/latex/microtype/microtype-pdftex.def
## +File: microtype-pdftex.def 2019/11/18 v2.7d Definitions specific to pdftex (RS)
## +
## +LaTeX Info: Redefining \lsstyle on input line 914.
## +LaTeX Info: Redefining \lslig on input line 914.
## +\MT@outer@space=\skip63
## +)
## +Package microtype Info: Loading configuration file microtype.cfg.
## +(/usr/local/texlive/2020/texmf-dist/tex/latex/microtype/microtype.cfg
## +File: microtype.cfg 2019/11/18 v2.7d microtype main configuration file (RS)
##  ))
## -\Gin@req@height=\dimen157
## -\Gin@req@width=\dimen158
## -) (/usr/local/texlive/2020/texmf-dist/tex/latex/grffile/grffile.sty
## +Package hyperref Info: Option `breaklinks' set `true' on input line 33.
## +(/usr/local/texlive/2020/texmf-dist/tex/latex/grffile/grffile.sty
##  Package: grffile 2019/11/11 v2.1 Extended file name support for graphics (legac
##  y)
##  Package grffile Info: This package is an empty stub for compatibility on input 
##  line 40.
## -) (/usr/local/texlive/2020/texmf-dist/tex/latex/parskip/parskip.sty
## -Package: parskip 2020-01-22 v2.0d non-zero parskip adjustments
## -(/usr/local/texlive/2020/texmf-dist/tex/latex/etoolbox/etoolbox.sty
## -Package: etoolbox 2019/09/21 v2.5h e-TeX tools for LaTeX (JAW)
## -\etb@tempcnta=\count268
## -)) (/usr/local/texlive/2020/texmf-dist/tex/latex/l3backend/l3backend-pdfmode.de
## -f
## +) (/usr/local/texlive/2020/texmf-dist/tex/latex/l3backend/l3backend-pdfmode.def
##  File: l3backend-pdfmode.def 2020-03-12 L3 backend support: PDF mode
## -\l__kernel_color_stack_int=\count269
## -\l__pdf_internal_box=\box47
## +\l__kernel_color_stack_int=\count312
## +\l__pdf_internal_box=\box73
##  )
##  No file SessionPlan.aux.
##  \openout1 = `SessionPlan.aux'.
##  
## -LaTeX Font Info:    Checking defaults for OML/cmm/m/it on input line 72.
## -LaTeX Font Info:    ... okay on input line 72.
## -LaTeX Font Info:    Checking defaults for OMS/cmsy/m/n on input line 72.
## -LaTeX Font Info:    ... okay on input line 72.
## -LaTeX Font Info:    Checking defaults for OT1/cmr/m/n on input line 72.
## -LaTeX Font Info:    ... okay on input line 72.
## -LaTeX Font Info:    Checking defaults for T1/cmr/m/n on input line 72.
## -LaTeX Font Info:    ... okay on input line 72.
## -LaTeX Font Info:    Checking defaults for TS1/cmr/m/n on input line 72.
## -LaTeX Font Info:    ... okay on input line 72.
## -LaTeX Font Info:    Checking defaults for OMX/cmex/m/n on input line 72.
## -LaTeX Font Info:    ... okay on input line 72.
## -LaTeX Font Info:    Checking defaults for U/cmr/m/n on input line 72.
## -LaTeX Font Info:    ... okay on input line 72.
## -LaTeX Font Info:    Checking defaults for PD1/pdf/m/n on input line 72.
## -LaTeX Font Info:    ... okay on input line 72.
## -LaTeX Font Info:    Checking defaults for PU/pdf/m/n on input line 72.
## -LaTeX Font Info:    ... okay on input line 72.
## -LaTeX Info: Redefining \microtypecontext on input line 72.
## -Package microtype Info: Generating PDF output.
## -Package microtype Info: Character protrusion enabled (level 2).
## -Package microtype Info: Using protrusion set `basicmath'.
## -Package microtype Info: Automatic font expansion enabled (level 2),
## -(microtype)             stretch: 20, shrink: 20, step: 1, non-selected.
## -Package microtype Info: Using default expansion set `basictext'.
## -LaTeX Info: Redefining \showhyphens on input line 72.
## -Package microtype Info: No adjustment of tracking.
## -Package microtype Info: No adjustment of interword spacing.
## -Package microtype Info: No adjustment of character kerning.
## -(/usr/local/texlive/2020/texmf-dist/tex/latex/microtype/mt-cmr.cfg
## -File: mt-cmr.cfg 2013/05/19 v2.2 microtype config. file: Computer Modern Roman 
## -(RS)
## -)
## -\AtBeginShipoutBox=\box48
## -Package hyperref Info: Link coloring OFF on input line 72.
## -(/usr/local/texlive/2020/texmf-dist/tex/latex/hyperref/nameref.sty
## -Package: nameref 2019/09/16 v2.46 Cross-referencing by name of section
## -(/usr/local/texlive/2020/texmf-dist/tex/latex/refcount/refcount.sty
## -Package: refcount 2019/12/15 v3.6 Data extraction from label references (HO)
## -)
## -(/usr/local/texlive/2020/texmf-dist/tex/generic/gettitlestring/gettitlestring.s
## -ty
## -Package: gettitlestring 2019/12/15 v1.6 Cleanup title references (HO)
## -)
## -\c@section@level=\count270
## -)
## -LaTeX Info: Redefining \ref on input line 72.
## -LaTeX Info: Redefining \pageref on input line 72.
## -LaTeX Info: Redefining \nameref on input line 72.
## -\@outlinefile=\write3
## -\openout3 = `SessionPlan.out'.
## -
## +LaTeX Font Info:    Checking defaults for OML/cmm/m/it on input line 79.
## +LaTeX Font Info:    ... okay on input line 79.
## +LaTeX Font Info:    Checking defaults for OMS/cmsy/m/n on input line 79.
## +LaTeX Font Info:    ... okay on input line 79.
## +LaTeX Font Info:    Checking defaults for OT1/cmr/m/n on input line 79.
## +LaTeX Font Info:    ... okay on input line 79.
## +LaTeX Font Info:    Checking defaults for T1/cmr/m/n on input line 79.
## +LaTeX Font Info:    ... okay on input line 79.
## +LaTeX Font Info:    Checking defaults for TS1/cmr/m/n on input line 79.
## +LaTeX Font Info:    ... okay on input line 79.
## +LaTeX Font Info:    Checking defaults for OMX/cmex/m/n on input line 79.
## +LaTeX Font Info:    ... okay on input line 79.
## +LaTeX Font Info:    Checking defaults for U/cmr/m/n on input line 79.
## +LaTeX Font Info:    ... okay on input line 79.
## +LaTeX Font Info:    Checking defaults for PD1/pdf/m/n on input line 79.
## +LaTeX Font Info:    ... okay on input line 79.
##  *geometry* driver: auto-detecting
##  *geometry* detected driver: pdftex
##  *geometry* verbose mode - [ preamble ] result:
##  * driver: pdftex
## -* paper: <default>
## +* paper: custom
##  * layout: <same size as paper>
##  * layoutoffset:(h,v)=(0.0pt,0.0pt)
## -* modes: 
## -* h-part:(L,W,R)=(72.26999pt, 469.75502pt, 72.26999pt)
## -* v-part:(T,H,B)=(72.26999pt, 650.43001pt, 72.26999pt)
## -* \paperwidth=614.295pt
## -* \paperheight=794.96999pt
## -* \textwidth=469.75502pt
## -* \textheight=650.43001pt
## -* \oddsidemargin=0.0pt
## -* \evensidemargin=0.0pt
## -* \topmargin=-37.0pt
## -* \headheight=12.0pt
## -* \headsep=25.0pt
## -* \topskip=10.0pt
## -* \footskip=30.0pt
## -* \marginparwidth=65.0pt
## -* \marginparsep=11.0pt
## +* modes: includehead includefoot 
## +* h-part:(L,W,R)=(28.45274pt, 307.28987pt, 28.45274pt)
## +* v-part:(T,H,B)=(0.0pt, 273.14662pt, 0.0pt)
## +* \paperwidth=364.19536pt
## +* \paperheight=273.14662pt
## +* \textwidth=307.28987pt
## +* \textheight=244.6939pt
## +* \oddsidemargin=-43.81725pt
## +* \evensidemargin=-43.81725pt
## +* \topmargin=-72.26999pt
## +* \headheight=14.22636pt
## +* \headsep=0.0pt
## +* \topskip=11.0pt
## +* \footskip=14.22636pt
## +* \marginparwidth=4.0pt
## +* \marginparsep=10.0pt
##  * \columnsep=10.0pt
## -* \skip\footins=9.0pt plus 4.0pt minus 2.0pt
## +* \skip\footins=10.0pt plus 4.0pt minus 2.0pt
##  * \hoffset=0.0pt
##  * \voffset=0.0pt
##  * \mag=1000
## @@ -418,17 +752,17 @@ LaTeX Info: Redefining \nameref on input line 72.
##  
##  (/usr/local/texlive/2020/texmf-dist/tex/context/base/mkii/supp-pdf.mkii
##  [Loading MPS to PDF converter (version 2006.09.02).]
## -\scratchcounter=\count271
## -\scratchdimen=\dimen159
## -\scratchbox=\box49
## -\nofMPsegments=\count272
## -\nofMParguments=\count273
## -\everyMPshowfont=\toks26
## -\MPscratchCnt=\count274
## -\MPscratchDim=\dimen160
## -\MPnumerator=\count275
## -\makeMPintoPDFobject=\count276
## -\everyMPtoPDFconversion=\toks27
## +\scratchcounter=\count313
## +\scratchdimen=\dimen278
## +\scratchbox=\box74
## +\nofMPsegments=\count314
## +\nofMParguments=\count315
## +\everyMPshowfont=\toks47
## +\MPscratchCnt=\count316
## +\MPscratchDim=\dimen279
## +\MPnumerator=\count317
## +\makeMPintoPDFobject=\count318
## +\everyMPtoPDFconversion=\toks48
##  ) (/usr/local/texlive/2020/texmf-dist/tex/latex/epstopdf-pkg/epstopdf-base.sty
##  Package: epstopdf-base 2020-01-24 v2.11 Base part for package epstopdf
##  Package epstopdf-base Info: Redefining graphics rule for `.eps' on input line 4
## @@ -436,224 +770,420 @@ Package epstopdf-base Info: Redefining graphics rule for `.eps' on input line 4
##  (/usr/local/texlive/2020/texmf-dist/tex/latex/latexconfig/epstopdf-sys.cfg
##  File: epstopdf-sys.cfg 2010/07/13 v1.3 Configuration of (r)epstopdf for TeX Liv
##  e
## -))
## -LaTeX Font Info:    Trying to load font information for OT1+lmr on input line 7
## -4.
## -(/usr/local/texlive/2020/texmf-dist/tex/latex/lm/ot1lmr.fd
## -File: ot1lmr.fd 2009/10/30 v1.6 Font defs for Latin Modern
## +)) ABD: EveryShipout initializing macros
## +\AtBeginShipoutBox=\box75
## +Package hyperref Info: Link coloring OFF on input line 79.
## +(/usr/local/texlive/2020/texmf-dist/tex/latex/hyperref/nameref.sty
## +Package: nameref 2019/09/16 v2.46 Cross-referencing by name of section
## +(/usr/local/texlive/2020/texmf-dist/tex/latex/refcount/refcount.sty
## +Package: refcount 2019/12/15 v3.6 Data extraction from label references (HO)
## +)
## +(/usr/local/texlive/2020/texmf-dist/tex/generic/gettitlestring/gettitlestring.s
## +ty
## +Package: gettitlestring 2019/12/15 v1.6 Cleanup title references (HO)
## +)
## +\c@section@level=\count319
## +)
## +LaTeX Info: Redefining \ref on input line 79.
## +LaTeX Info: Redefining \pageref on input line 79.
## +LaTeX Info: Redefining \nameref on input line 79.
## +\@outlinefile=\write5
## +\openout5 = `SessionPlan.out'.
## +
## +LaTeX Font Info:    Overwriting symbol font `operators' in version `normal'
## +(Font)                  OT1/lmr/m/n --> OT1/cmss/m/n on input line 79.
## +LaTeX Font Info:    Overwriting symbol font `operators' in version `bold'
## +(Font)                  OT1/lmr/bx/n --> OT1/cmss/b/n on input line 79.
## +LaTeX Font Info:    Overwriting symbol font `operators' in version `normal'
## +(Font)                  OT1/cmss/m/n --> OT1/lmss/m/n on input line 79.
## +LaTeX Font Info:    Overwriting symbol font `operators' in version `bold'
## +(Font)                  OT1/cmss/b/n --> OT1/lmss/b/n on input line 79.
## +\symnumbers=\mathgroup6
## +\sympureletters=\mathgroup7
## +LaTeX Font Info:    Overwriting math alphabet `\mathrm' in version `normal'
## +(Font)                  OT1/lmss/m/n --> T1/lmr/m/n on input line 79.
## +LaTeX Font Info:    Redeclaring math alphabet \mathbf on input line 79.
## +LaTeX Font Info:    Overwriting math alphabet `\mathbf' in version `normal'
## +(Font)                  OT1/lmr/bx/n --> T1/lmss/b/n on input line 79.
## +LaTeX Font Info:    Overwriting math alphabet `\mathbf' in version `bold'
## +(Font)                  OT1/lmr/bx/n --> T1/lmss/b/n on input line 79.
## +LaTeX Font Info:    Redeclaring math alphabet \mathsf on input line 79.
## +LaTeX Font Info:    Overwriting math alphabet `\mathsf' in version `normal'
## +(Font)                  OT1/lmss/m/n --> T1/lmss/m/n on input line 79.
## +LaTeX Font Info:    Overwriting math alphabet `\mathsf' in version `bold'
## +(Font)                  OT1/lmss/bx/n --> T1/lmss/m/n on input line 79.
## +LaTeX Font Info:    Redeclaring math alphabet \mathit on input line 79.
## +LaTeX Font Info:    Overwriting math alphabet `\mathit' in version `normal'
## +(Font)                  OT1/lmr/m/it --> T1/lmss/m/it on input line 79.
## +LaTeX Font Info:    Overwriting math alphabet `\mathit' in version `bold'
## +(Font)                  OT1/lmr/bx/it --> T1/lmss/m/it on input line 79.
## +LaTeX Font Info:    Redeclaring math alphabet \mathtt on input line 79.
## +LaTeX Font Info:    Overwriting math alphabet `\mathtt' in version `normal'
## +(Font)                  OT1/lmtt/m/n --> T1/lmtt/m/n on input line 79.
## +LaTeX Font Info:    Overwriting math alphabet `\mathtt' in version `bold'
## +(Font)                  OT1/lmtt/m/n --> T1/lmtt/m/n on input line 79.
## +LaTeX Font Info:    Overwriting symbol font `numbers' in version `bold'
## +(Font)                  T1/lmss/m/n --> T1/lmss/b/n on input line 79.
## +LaTeX Font Info:    Overwriting symbol font `pureletters' in version `bold'
## +(Font)                  T1/lmss/m/it --> T1/lmss/b/it on input line 79.
## +LaTeX Font Info:    Overwriting math alphabet `\mathrm' in version `bold'
## +(Font)                  OT1/lmss/b/n --> T1/lmr/b/n on input line 79.
## +LaTeX Font Info:    Overwriting math alphabet `\mathbf' in version `bold'
## +(Font)                  T1/lmss/b/n --> T1/lmss/b/n on input line 79.
## +LaTeX Font Info:    Overwriting math alphabet `\mathsf' in version `bold'
## +(Font)                  T1/lmss/m/n --> T1/lmss/b/n on input line 79.
## +LaTeX Font Info:    Overwriting math alphabet `\mathit' in version `bold'
## +(Font)                  T1/lmss/m/it --> T1/lmss/b/it on input line 79.
## +LaTeX Font Info:    Overwriting math alphabet `\mathtt' in version `bold'
## +(Font)                  T1/lmtt/m/n --> T1/lmtt/b/n on input line 79.
## +
## +(/usr/local/texlive/2020/texmf-dist/tex/latex/translator/translator-basic-dicti
## +onary-English.dict
## +Dictionary: translator-basic-dictionary, Language: English 
## +)
## +(/usr/local/texlive/2020/texmf-dist/tex/latex/translator/translator-bibliograph
## +y-dictionary-English.dict
## +Dictionary: translator-bibliography-dictionary, Language: English 
## +)
## +(/usr/local/texlive/2020/texmf-dist/tex/latex/translator/translator-environment
## +-dictionary-English.dict
## +Dictionary: translator-environment-dictionary, Language: English 
##  )
## -LaTeX Font Info:    Trying to load font information for OML+lmm on input line 7
## -4.
## +(/usr/local/texlive/2020/texmf-dist/tex/latex/translator/translator-months-dict
## +ionary-English.dict
## +Dictionary: translator-months-dictionary, Language: English 
## +)
## +(/usr/local/texlive/2020/texmf-dist/tex/latex/translator/translator-numbers-dic
## +tionary-English.dict
## +Dictionary: translator-numbers-dictionary, Language: English 
## +)
## +(/usr/local/texlive/2020/texmf-dist/tex/latex/translator/translator-theorem-dic
## +tionary-English.dict
## +Dictionary: translator-theorem-dictionary, Language: English 
## +)
## +LaTeX Info: Redefining \microtypecontext on input line 79.
## +Package microtype Info: Generating PDF output.
## +Package microtype Info: Character protrusion enabled (level 2).
## +Package microtype Info: Using protrusion set `basicmath'.
## +Package microtype Info: Automatic font expansion enabled (level 2),
## +(microtype)             stretch: 20, shrink: 20, step: 1, non-selected.
## +Package microtype Info: Using default expansion set `basictext'.
## +LaTeX Info: Redefining \showhyphens on input line 79.
## +Package microtype Info: No adjustment of tracking.
## +Package microtype Info: No adjustment of interword spacing.
## +Package microtype Info: No adjustment of character kerning.
## +Package microtype Info: Loading generic protrusion settings for font family
## +(microtype)             `lmss' (encoding: T1).
## +(microtype)             For optimal results, create family-specific settings.
## +(microtype)             See the microtype manual for details.
## +No file SessionPlan.nav.
## +[1
## +
## +{/usr/local/texlive/2020/texmf-var/fonts/map/pdftex/updmap/pdftex.map}] [2
## +
## +] [3
## +
## +]
## +\openout4 = `SessionPlan.vrb'.
## +
## +(./SessionPlan.vrb
## +LaTeX Font Info:    Trying to load font information for T1+lmtt on input line 3
## +.
## +(/usr/local/texlive/2020/texmf-dist/tex/latex/lm/t1lmtt.fd
## +File: t1lmtt.fd 2009/10/30 v1.6 Font defs for Latin Modern
## +)
## +Overfull \hbox (37.63414pt too wide) in paragraph at lines 7--7
## +[]\T1/lmtt/m/n/10.95 ## /Users/tarahenechowicz/Documents/RTeaching/IntroGitGitH
## +ub[] 
## + []
## +
## +
## +Overfull \hbox (20.38794pt too wide) in paragraph at lines 7--7
## +[]\T1/lmtt/m/n/10.95 ## bash: line 7: cd: Documents: No such file or directory[
## +] 
## + []
## +
## +) [4
## +
## +]
## +\openout4 = `SessionPlan.vrb'.
## +
## +(./SessionPlan.vrb
## +Overfull \hbox (307.82462pt too wide) in paragraph at lines 6--6
## +[]\T1/lmtt/m/n/10.95 ## Reinitialized existing Git repository in /Users/tarahen
## +echowicz/Documents/RTeaching/IntroGitGitHub/.git/[] 
## + []
## +
## +) [5
## +
## +]
## +LaTeX Font Info:    Trying to load font information for OT1+lmss on input line 
## +162.
## +(/usr/local/texlive/2020/texmf-dist/tex/latex/lm/ot1lmss.fd
## +File: ot1lmss.fd 2009/10/30 v1.6 Font defs for Latin Modern
## +)
## +Package microtype Info: Loading generic protrusion settings for font family
## +(microtype)             `lmss' (encoding: OT1).
## +(microtype)             For optimal results, create family-specific settings.
## +(microtype)             See the microtype manual for details.
## +LaTeX Font Info:    Trying to load font information for OML+lmm on input line 1
## +62.
##  (/usr/local/texlive/2020/texmf-dist/tex/latex/lm/omllmm.fd
##  File: omllmm.fd 2009/10/30 v1.6 Font defs for Latin Modern
##  )
##  LaTeX Font Info:    Trying to load font information for OMS+lmsy on input line 
## -74.
## +162.
##  (/usr/local/texlive/2020/texmf-dist/tex/latex/lm/omslmsy.fd
##  File: omslmsy.fd 2009/10/30 v1.6 Font defs for Latin Modern
##  )
##  LaTeX Font Info:    Trying to load font information for OMX+lmex on input line 
## -74.
## +162.
##  (/usr/local/texlive/2020/texmf-dist/tex/latex/lm/omxlmex.fd
##  File: omxlmex.fd 2009/10/30 v1.6 Font defs for Latin Modern
##  )
##  LaTeX Font Info:    External font `lmex10' loaded for size
## -(Font)              <12> on input line 74.
## +(Font)              <10.95> on input line 162.
##  LaTeX Font Info:    External font `lmex10' loaded for size
## -(Font)              <8> on input line 74.
## +(Font)              <8> on input line 162.
##  LaTeX Font Info:    External font `lmex10' loaded for size
## -(Font)              <6> on input line 74.
## -LaTeX Font Info:    Trying to load font information for U+msa on input line 74.
## -
## +(Font)              <6> on input line 162.
## +LaTeX Font Info:    Trying to load font information for U+msa on input line 162
## +.
##  (/usr/local/texlive/2020/texmf-dist/tex/latex/amsfonts/umsa.fd
##  File: umsa.fd 2013/01/14 v3.01 AMS symbols A
##  ) (/usr/local/texlive/2020/texmf-dist/tex/latex/microtype/mt-msa.cfg
##  File: mt-msa.cfg 2006/02/04 v1.1 microtype config. file: AMS symbols (a) (RS)
##  )
## -LaTeX Font Info:    Trying to load font information for U+msb on input line 74.
## -
## +LaTeX Font Info:    Trying to load font information for U+msb on input line 162
## +.
##  (/usr/local/texlive/2020/texmf-dist/tex/latex/amsfonts/umsb.fd
##  File: umsb.fd 2013/01/14 v3.01 AMS symbols B
##  ) (/usr/local/texlive/2020/texmf-dist/tex/latex/microtype/mt-msb.cfg
##  File: mt-msb.cfg 2005/06/01 v1.0 microtype config. file: AMS symbols (b) (RS)
##  )
## -LaTeX Font Info:    Trying to load font information for T1+lmtt on input line 1
## -00.
## -(/usr/local/texlive/2020/texmf-dist/tex/latex/lm/t1lmtt.fd
## -File: t1lmtt.fd 2009/10/30 v1.6 Font defs for Latin Modern
## -)
## -Overfull \hbox (91.99335pt too wide) in paragraph at lines 112--112
## -[]\T1/lmtt/m/n/10 ## Reinitialized existing Git repository in /Users/tarahenech
## -owicz/Documents/RTeaching/IntroGitGitHub/.git/[] 
## - []
## +LaTeX Font Info:    Font shape `T1/lmss/m/it' in size <10.95> not available
## +(Font)              Font shape `T1/lmss/m/sl' tried instead on input line 162.
## +LaTeX Font Info:    Font shape `T1/lmss/m/it' in size <8> not available
## +(Font)              Font shape `T1/lmss/m/sl' tried instead on input line 162.
## +LaTeX Font Info:    Font shape `T1/lmss/m/it' in size <6> not available
## +(Font)              Font shape `T1/lmss/m/sl' tried instead on input line 162.
## +[6
##  
## -LaTeX Font Info:    Trying to load font information for TS1+lmr on input line 1
## -19.
## -(/usr/local/texlive/2020/texmf-dist/tex/latex/lm/ts1lmr.fd
## -File: ts1lmr.fd 2009/10/30 v1.6 Font defs for Latin Modern
## -) [1
## +]
## +\openout4 = `SessionPlan.vrb'.
##  
## -{/usr/local/texlive/2020/texmf-var/fonts/map/pdftex/updmap/pdftex.map}]
## -Overfull \hbox (70.99341pt too wide) in paragraph at lines 457--457
## -[]\T1/lmtt/m/n/10 ## diff --git a/.Rproj.user/665D533C/sources/prop/E958F836 b/
## -.Rproj.user/665D533C/sources/prop/E958F836[] 
## +(./SessionPlan.vrb
## +Overfull \hbox (54.88034pt too wide) in paragraph at lines 11--11
## +[]\T1/lmtt/m/n/10.95 ##   (use "git add <file>..." to update what will be commi
## +tted)[] 
##   []
##  
## -[2]
## -LaTeX Font Info:    Trying to load font information for TS1+lmtt on input line 
## -457.
## -(/usr/local/texlive/2020/texmf-dist/tex/latex/lm/ts1lmtt.fd
## -File: ts1lmtt.fd 2009/10/30 v1.6 Font defs for Latin Modern
## -)
## -Overfull \hbox (55.24345pt too wide) in paragraph at lines 457--457
## -[]\T1/lmtt/m/n/10 ## @@ -80,7 +82,7 @@ you can also use \TS1/lmtt/m/n/10 `\T1/l
## -mtt/m/n/10 git add -A\TS1/lmtt/m/n/10 ` \T1/lmtt/m/n/10 to add all the files in
## - your repository (remember[] 
## +
## +Overfull \hbox (118.11641pt too wide) in paragraph at lines 11--11
## +[]\T1/lmtt/m/n/10.95 ##   (use "git restore <file>..." to discard changes in wo
## +rking directory)[] 
##   []
##  
##  
## -Overfull \hbox (34.24352pt too wide) in paragraph at lines 457--457
## -[]\T1/lmtt/m/n/10 ##  Let\TS1/lmtt/m/n/10 '\T1/lmtt/m/n/10 s save our first set
## - of changes using a commit. You need to put a message with a commit.[] 
## +Overfull \hbox (83.62401pt too wide) in paragraph at lines 11--11
## +[]\T1/lmtt/m/n/10.95 ## no changes added to commit (use "git add" and/or "git c
## +ommit -a")[] 
##   []
##  
## +) [7
## +
## +]
## +\openout4 = `SessionPlan.vrb'.
##  
## -Overfull \hbox (18.49356pt too wide) in paragraph at lines 457--457
## -[]\T1/lmtt/m/n/10 ## -1.Create a repository on github that matches the name of 
## -your repository on your computer[] 
## +(./SessionPlan.vrb) [8
## +
## +] [9
## +
## +]
## +\openout4 = `SessionPlan.vrb'.
## +
## +(./SessionPlan.vrb
## +Overfull \hbox (72.12654pt too wide) in paragraph at lines 42--42
## +[]\T1/lmtt/m/n/10.95 ##   (use "git add/rm <file>..." to update what will be co
## +mmitted)[] 
##   []
##  
##  
## -Overfull \hbox (286.24278pt too wide) in paragraph at lines 457--457
## -[]\T1/lmtt/m/n/10 ## -- take our repository from our computer and "push it" to 
## -github so it can be store on their server in addition to or instead of our comp
## -uter[] 
## +Overfull \hbox (118.11641pt too wide) in paragraph at lines 42--42
## +[]\T1/lmtt/m/n/10.95 ##   (use "git restore <file>..." to discard changes in wo
## +rking directory)[] 
##   []
##  
##  
## -Overfull \hbox (7.99359pt too wide) in paragraph at lines 457--457
## -[]\T1/lmtt/m/n/10 ## +Create a repository on github that matches the name of yo
## -ur repository on your computer[] 
## +Overfull \hbox (77.87527pt too wide) in paragraph at lines 42--42
## +[]\T1/lmtt/m/n/10.95 ##   (use "git add <file>..." to include in what will be c
## +ommitted)[] 
##   []
##  
##  
## -Overfull \hbox (354.49258pt too wide) in paragraph at lines 457--457
## -[]\T1/lmtt/m/n/10 ##    3. now that its on github your team members or even gen
## -eral public can collaborate with you on your work and you can log the changes o
## -f multiple users![] 
## +Overfull \hbox (83.62401pt too wide) in paragraph at lines 42--42
## +[]\T1/lmtt/m/n/10.95 ## no changes added to commit (use "git add" and/or "git c
## +ommit -a")[] 
## + []
## +
## +LaTeX Font Info:    Trying to load font information for TS1+lmtt on input line 
## +42.
## +(/usr/local/texlive/2020/texmf-dist/tex/latex/lm/ts1lmtt.fd
## +File: ts1lmtt.fd 2009/10/30 v1.6 Font defs for Latin Modern
## +)
## +Overfull \hbox (3.14174pt too wide) in paragraph at lines 42--42
## +[]\T1/lmtt/m/n/10.95 ## Binary files a/SessionPlan.pdf and /dev/null differ[] 
##   []
##  
##  
## -Overfull \hbox (291.49277pt too wide) in paragraph at lines 457--457
## -[]\T1/lmtt/m/n/10 ## -To push our repository to github, we need to first set up
## - the connection between the repository on our computer with the repository on g
## -ithub[] 
## +Overfull \hbox (3.14174pt too wide) in paragraph at lines 42--42
## +[]\T1/lmtt/m/n/10.95 ##   (use "git restore --staged <file>..." to unstage)[] 
## + []
## +
## +)
## +Overfull \vbox (278.5271pt too high) detected at line 246
##   []
##  
## -[3]
## -Overfull \hbox (296.74275pt too wide) in paragraph at lines 457--457
## -[]\T1/lmtt/m/n/10 ## +-To push our repository to github, we need to first set u
## -p the connection between the repository on our computer with the repository on 
## -github[] 
## +[10
## +
## +]
## +\openout4 = `SessionPlan.vrb'.
## +
## +(./SessionPlan.vrb
## +Overfull \hbox (3.14174pt too wide) in paragraph at lines 9--9
## +[]\T1/lmtt/m/n/10.95 ##  4 files changed, 1468 insertions(+), 1 deletion(-)[] 
##   []
##  
## +) [11
## +
## +]
## +\openout4 = `SessionPlan.vrb'.
##  
## -Overfull \hbox (165.49313pt too wide) in paragraph at lines 457--457
## -[]\T1/lmtt/m/n/10 ## +- go to the repository that we just made and there is the
## - green button called "code" and copy the code from the https[] 
## +(./SessionPlan.vrb
## +Overfull \hbox (49.1316pt too wide) in paragraph at lines 81--81
## +[]\T1/lmtt/m/n/10.95 ## Author: tarahenechowicz <tara.henechowicz@mail.utoronto
## +.ca>[] 
##   []
##  
##  
## -Overfull \hbox (160.24315pt too wide) in paragraph at lines 457--457
## -[]\T1/lmtt/m/n/10 ## -#go to the repository that we just made and there is the 
## -green button called "code" and copy the code from the https[] 
## +Overfull \hbox (49.1316pt too wide) in paragraph at lines 81--81
## +[]\T1/lmtt/m/n/10.95 ## Author: tarahenechowicz <tara.henechowicz@mail.utoronto
## +.ca>[] 
##   []
##  
##  
## -Overfull \hbox (23.74355pt too wide) in paragraph at lines 457--457
## -[]\T1/lmtt/m/n/10 ## +- GitHub no longer takes your account password, you need 
## -to set up an authentication token[] 
## +Overfull \hbox (49.1316pt too wide) in paragraph at lines 81--81
## +[]\T1/lmtt/m/n/10.95 ## Author: tarahenechowicz <tara.henechowicz@mail.utoronto
## +.ca>[] 
##   []
##  
##  
## -Overfull \hbox (317.74269pt too wide) in paragraph at lines 457--457
## -[]\T1/lmtt/m/n/10 ##  Git authetication token start up : https://docs.github.co
## -m/en/authentication/keeping-your-account-and-data-secure/creating-a-personal-ac
## -cess-token[] 
## +Overfull \hbox (49.1316pt too wide) in paragraph at lines 81--81
## +[]\T1/lmtt/m/n/10.95 ## Author: tarahenechowicz <tara.henechowicz@mail.utoronto
## +.ca>[] 
##   []
##  
## -[4]
## -Overfull \hbox (112.99329pt too wide) in paragraph at lines 457--457
## -[]\T1/lmtt/m/n/10 ## -  #when there is a divergence or a branch, it takes all t
## -he divergence and squish it back into the original[] 
## +
## +Overfull \hbox (49.1316pt too wide) in paragraph at lines 81--81
## +[]\T1/lmtt/m/n/10.95 ## Author: tarahenechowicz <tara.henechowicz@mail.utoronto
## +.ca>[] 
##   []
##  
##  
## -Overfull \hbox (2.7436pt too wide) in paragraph at lines 457--457
## -[]\T1/lmtt/m/n/10 ## -#just an example of the many wyas github is helpful for c
## -ollaborating with scientists.[] 
## +Overfull \hbox (49.1316pt too wide) in paragraph at lines 81--81
## +[]\T1/lmtt/m/n/10.95 ## Author: tarahenechowicz <tara.henechowicz@mail.utoronto
## +.ca>[] 
##   []
##  
##  
## -Overfull \hbox (55.24345pt too wide) in paragraph at lines 457--457
## -[]\T1/lmtt/m/n/10 ## +- Issues are like little emails that you can send to your
## -self on GitHub or tag and assign others[] 
## +Overfull \hbox (49.1316pt too wide) in paragraph at lines 81--81
## +[]\T1/lmtt/m/n/10.95 ## Author: tarahenechowicz <tara.henechowicz@mail.utoronto
## +.ca>[] 
##   []
##  
##  
## -Overfull \hbox (65.74342pt too wide) in paragraph at lines 457--457
## -[]\T1/lmtt/m/n/10 ## +Let\TS1/lmtt/m/n/10 '\T1/lmtt/m/n/10 s try to make an iss
## -ue called "Set up github profile" on GitHub with the following to-do list:[] 
## +Overfull \hbox (49.1316pt too wide) in paragraph at lines 81--81
## +[]\T1/lmtt/m/n/10.95 ## Author: tarahenechowicz <tara.henechowicz@mail.utoronto
## +.ca>[] 
##   []
##  
##  
## -Overfull \hbox (44.74348pt too wide) in paragraph at lines 457--457
## -[]\T1/lmtt/m/n/10 ## +- Helpful for version control and collaboration if you wa
## -nt multiple versions of the same work[] 
## +Overfull \hbox (49.1316pt too wide) in paragraph at lines 81--81
## +[]\T1/lmtt/m/n/10.95 ## Author: tarahenechowicz <tara.henechowicz@mail.utoronto
## +.ca>[] 
##   []
##  
##  
## -Overfull \hbox (49.99347pt too wide) in paragraph at lines 457--457
## -[]\T1/lmtt/m/n/10 ## +- This is an introduction to a complicated git topic - th
## -e more branches, the trickier it gets.[] 
## +Overfull \hbox (54.88034pt too wide) in paragraph at lines 81--81
## +[]\T1/lmtt/m/n/10.95 ## Author: Tara Henechowicz <tara.henechowicz@mail.utoront
## +o.ca>[] 
##   []
##  
## -[5]
## -Overfull \hbox (401.74245pt too wide) in paragraph at lines 457--457
## -[]\T1/lmtt/m/n/10 ##  #allows you to split off onto a new version of work witho
## -ut affecting the main branch. Helpful for version control if you want multiple 
## -versions of the same work.[] 
## +
## +Overfull \hbox (54.88034pt too wide) in paragraph at lines 81--81
## +[]\T1/lmtt/m/n/10.95 ## Author: Tara Henechowicz <tara.henechowicz@mail.utoront
## +o.ca>[] 
##   []
##  
##  
## -Overfull \hbox (39.4935pt too wide) in paragraph at lines 457--457
## -[]\T1/lmtt/m/n/10 ## +- get\TS1/lmtt/m/n/10 '\T1/lmtt/m/n/10 s more complicated
## - with more complicated changes, code, or with multiple branches/users[] 
## +Overfull \hbox (54.88034pt too wide) in paragraph at lines 81--81
## +[]\T1/lmtt/m/n/10.95 ## Author: Tara Henechowicz <tara.henechowicz@mail.utoront
## +o.ca>[] 
##   []
##  
##  
## -Overfull \hbox (7.99359pt too wide) in paragraph at lines 457--457
## -[]\T1/lmtt/m/n/10 ## +- try forking one of my repos from: https://github.com/ta
## -rahenechowicz?tab=repositories[] 
## +Overfull \hbox (54.88034pt too wide) in paragraph at lines 81--81
## +[]\T1/lmtt/m/n/10.95 ## Author: Tara Henechowicz <tara.henechowicz@mail.utoront
## +o.ca>[] 
##   []
##  
## -[6]
## -Overfull \hbox (49.99347pt too wide) in paragraph at lines 457--457
## -[]\T1/lmtt/m/n/10 ##  https://github.com/tarahenechowicz/studyGroup/blob/gh-pag
## -es/lessons/git/collaboration/lesson.md[] 
## +)
## +Overfull \vbox (808.92734pt too high) detected at line 342
##   []
##  
## -[7] [8]
## +[12
## +
## +] [13
## +
## +] [14
## +
## +]
## +\openout4 = `SessionPlan.vrb'.
## +
## +(./SessionPlan.vrb) [15
## +
## +]
##  LaTeX Font Info:    External font `lmex10' loaded for size
## -(Font)              <10> on input line 579.
## +(Font)              <10> on input line 415.
##  LaTeX Font Info:    External font `lmex10' loaded for size
## -(Font)              <7> on input line 579.
## +(Font)              <7> on input line 415.
##  LaTeX Font Info:    External font `lmex10' loaded for size
## -(Font)              <5> on input line 579.
## -[9]
## +(Font)              <5> on input line 415.
## +LaTeX Font Info:    Font shape `T1/lmss/m/it' in size <10> not available
## +(Font)              Font shape `T1/lmss/m/sl' tried instead on input line 415.
## +LaTeX Font Info:    Font shape `T1/lmss/m/it' in size <7> not available
## +(Font)              Font shape `T1/lmss/m/sl' tried instead on input line 415.
## +LaTeX Font Info:    Font shape `T1/lmss/m/it' in size <5> not available
## +(Font)              Font shape `T1/lmss/m/sl' tried instead on input line 415.
## +[16
## +
## +]
## +\openout4 = `SessionPlan.vrb'.
## +
## +(./SessionPlan.vrb) [17
## +
## +] [18
## +
## +] [19
## +
## +] [20
## +
## +] [21
## +
## +]
##  
##  LaTeX Warning: File `https://www.nobledesktop.com/image/gitresources/git-branch
## -es-merge.png' not found on input line 696.
## +es-merge.png' not found on input line 519.
##  
##  
##  ! Package pdftex.def Error: File `https://www.nobledesktop.com/image/gitresourc
## @@ -663,15 +1193,15 @@ See the pdftex.def package documentation for explanation.
##  Type  H <return>  for immediate help.
##   ...                                              
##                                                    
## -l.696 ...mage/gitresources/git-branches-merge.png}
## -                                                   
## +l.519 \end{frame}
## +                  
##  Here is how much of TeX's memory you used:
## - 12314 strings out of 480608
## - 183124 string characters out of 5903631
## - 466228 words of memory out of 5000000
## - 27919 multiletter control sequences out of 15000+600000
## - 576796 words of font info for 78 fonts, out of 8000000 for 9000
## + 19827 strings out of 480608
## + 377733 string characters out of 5903631
## + 657440 words of memory out of 5000000
## + 35110 multiletter control sequences out of 15000+600000
## + 597395 words of font info for 90 fonts, out of 8000000 for 9000
##   1141 hyphenation exceptions out of 8191
## - 34i,6n,42p,974b,468s stack positions out of 5000i,500n,10000p,200000b,80000s
## + 58i,11n,67p,815b,458s stack positions out of 5000i,500n,10000p,200000b,80000s
##  
##  !  ==> Fatal error occurred, no output PDF file produced!
## diff --git a/SessionPlan.tex b/SessionPlan.tex
## index 333ae72..fdf9b64 100644
## --- a/SessionPlan.tex
## +++ b/SessionPlan.tex
## @@ -1,4 +1,8 @@
## -\documentclass[]{article}
## +\documentclass[ignorenonframetext,]{beamer}
## +\setbeamertemplate{caption}[numbered]
## +\setbeamertemplate{caption label separator}{: }
## +\setbeamercolor{caption name}{fg=normal text.fg}
## +\beamertemplatenavigationsymbolsempty
##  \usepackage{lmodern}
##  \usepackage{amssymb,amsmath}
##  \usepackage{ifxetex,ifluatex}
## @@ -18,61 +22,64 @@
##  \IfFileExists{upquote.sty}{\usepackage{upquote}}{}
##  % use microtype if available
##  \IfFileExists{microtype.sty}{%
## -\usepackage[]{microtype}
## +\usepackage{microtype}
##  \UseMicrotypeSet[protrusion]{basicmath} % disable protrusion for tt fonts
##  }{}
## -\PassOptionsToPackage{hyphens}{url} % url is loaded by hyperref
## -\usepackage[unicode=true]{hyperref}
## +\newif\ifbibliography
##  \hypersetup{
##              pdftitle={Intro to Git and GitHub},
##              pdfauthor={Tara Henechowicz},
##              pdfborder={0 0 0},
##              breaklinks=true}
##  \urlstyle{same}  % don't use monospace font for urls
## -\usepackage[margin=1in]{geometry}
##  \usepackage{graphicx,grffile}
##  \makeatletter
##  \def\maxwidth{\ifdim\Gin@nat@width>\linewidth\linewidth\else\Gin@nat@width\fi}
## -\def\maxheight{\ifdim\Gin@nat@height>\textheight\textheight\else\Gin@nat@height\fi}
## +\def\maxheight{\ifdim\Gin@nat@height>\textheight0.8\textheight\else\Gin@nat@height\fi}
##  \makeatother
##  % Scale images if necessary, so that they will not overflow the page
##  % margins by default, and it is still possible to overwrite the defaults
##  % using explicit options in \includegraphics[width, height, ...]{}
##  \setkeys{Gin}{width=\maxwidth,height=\maxheight,keepaspectratio}
## -\IfFileExists{parskip.sty}{%
## -\usepackage{parskip}
## -}{% else
## +
## +% Prevent slide breaks in the middle of a paragraph:
## +\widowpenalties 1 10000
## +\raggedbottom
## +
## +\AtBeginPart{
## +  \let\insertpartnumber\relax
## +  \let\partname\relax
## +  \frame{\partpage}
## +}
## +\AtBeginSection{
## +  \ifbibliography
## +  \else
## +    \let\insertsectionnumber\relax
## +    \let\sectionname\relax
## +    \frame{\sectionpage}
## +  \fi
## +}
## +\AtBeginSubsection{
## +  \let\insertsubsectionnumber\relax
## +  \let\subsectionname\relax
## +  \frame{\subsectionpage}
## +}
## +
##  \setlength{\parindent}{0pt}
##  \setlength{\parskip}{6pt plus 2pt minus 1pt}
## -}
##  \setlength{\emergencystretch}{3em}  % prevent overfull lines
##  \providecommand{\tightlist}{%
##    \setlength{\itemsep}{0pt}\setlength{\parskip}{0pt}}
##  \setcounter{secnumdepth}{0}
## -% Redefines (sub)paragraphs to behave more like sections
## -\ifx\paragraph\undefined\else
## -\let\oldparagraph\paragraph
## -\renewcommand{\paragraph}[1]{\oldparagraph{#1}\mbox{}}
## -\fi
## -\ifx\subparagraph\undefined\else
## -\let\oldsubparagraph\subparagraph
## -\renewcommand{\subparagraph}[1]{\oldsubparagraph{#1}\mbox{}}
## -\fi
## -
## -% set default figure placement to htbp
## -\makeatletter
## -\def\fps@figure{htbp}
## -\makeatother
## -
##  
##  \title{Intro to Git and GitHub}
##  \author{Tara Henechowicz}
##  \date{April 25, 2022}
##  
##  \begin{document}
## -\maketitle
## +\frame{\titlepage}
##  
## -\section{Session Outline}\label{session-outline}
## +\begin{frame}{Session Outline}
##  
##  \begin{enumerate}
##  \def\labelenumi{\arabic{enumi}.}
## @@ -92,10 +99,14 @@
##    GitHub features for Collaboration
##  \end{enumerate}
##  
## -\section{Step 1. Configure git}\label{step-1.-configure-git}
## +\end{frame}
## +
## +\begin{frame}{Step 1. Configure git}
##  
## -\section{Step 2. Create a Repository on your computer with
## -gitGit}\label{step-2.-create-a-repository-on-your-computer-with-gitgit}
## +\end{frame}
## +
## +\begin{frame}[fragile]{Step 2. Create a Repository on your computer with
## +gitGit}
##  
##  \begin{verbatim}
##  ## /Users/tarahenechowicz/Documents/RTeaching/IntroGitGitHub
## @@ -103,15 +114,18 @@ gitGit}\label{step-2.-create-a-repository-on-your-computer-with-gitgit}
##  ## mkdir: GitPractice2: File exists
##  \end{verbatim}
##  
## -\section{Step 3. Git Init (Initialize the
## -Repo)}\label{step-3.-git-init-initialize-the-repo}
## +\end{frame}
## +
## +\begin{frame}[fragile]{Step 3. Git Init (Initialize the Repo)}
##  
##  \begin{verbatim}
##  ## Reinitialized existing Git repository in /Users/tarahenechowicz/Documents/RTeaching/IntroGitGitHub/.git/
##  ## master
##  \end{verbatim}
##  
## -\section{Adding files to the Repo}\label{adding-files-to-the-repo}
## +\end{frame}
## +
## +\begin{frame}{Adding files to the Repo}
##  
##  \begin{itemize}
##  \tightlist
## @@ -145,7 +159,9 @@ Repo)}\label{step-3.-git-init-initialize-the-repo}
##    Save the file.
##  \end{enumerate}
##  
## -\section{Making our first commit}\label{making-our-first-commit}
## +\end{frame}
## +
## +\begin{frame}[fragile]{Making our first commit}
##  
##  \begin{verbatim}
##  ## On branch master
## @@ -160,314 +176,102 @@ Repo)}\label{step-3.-git-init-initialize-the-repo}
##  you can also use \texttt{git\ add\ -A} to add all the files in your
##  repository (remember we only have one right now)
##  
## -\section{Making our first commit}\label{making-our-first-commit-1}
## +\end{frame}
## +
## +\begin{frame}[fragile]{Making our first commit}
##  
##  Let's save our first set of changes using a commit. You need to put a
##  message with a commit.
##  
##  \begin{verbatim}
## -## [master adc6a3d] Commit for the first time
## -##  1 file changed, 1 insertion(+), 1 deletion(-)
## +## [master 70df2d0] Commit for the first time
## +##  1 file changed, 1 insertion(+)
##  \end{verbatim}
##  
## -\section{Making our second commit}\label{making-our-second-commit}
## +\end{frame}
## +
## +\begin{frame}{Making our second commit}
##  
##  Now we are going to back to the text file and make a change. - for
##  example, I'm going to add my affiliation in the first line - you can add
##  your programming languages and experiences - or add a sentence about why
##  you want to use git
##  
## -\section{Check activity and changes with git status or git
## -diff}\label{check-activity-and-changes-with-git-status-or-git-diff}
## +\end{frame}
## +
## +\begin{frame}[fragile]{Check activity and changes with git status or git
## +diff}
##  
##  \begin{verbatim}
##  ## On branch master
##  ## Changes not staged for commit:
## -##   (use "git add <file>..." to update what will be committed)
## +##   (use "git add/rm <file>..." to update what will be committed)
##  ##   (use "git restore <file>..." to discard changes in working directory)
## -##  modified:   .Rproj.user/665D533C/sources/prop/E958F836
##  ##  modified:   SessionPlan.Rmd
## -##  modified:   SessionPlan.pdf
## +##  deleted:    SessionPlan.pdf
## +## 
## +## Untracked files:
## +##   (use "git add <file>..." to include in what will be committed)
## +##  SessionPlan.log
## +##  SessionPlan.tex
##  ## 
##  ## no changes added to commit (use "git add" and/or "git commit -a")
## -## diff --git a/.Rproj.user/665D533C/sources/prop/E958F836 b/.Rproj.user/665D533C/sources/prop/E958F836
## -## index 29af6e5..1c65c83 100644
## -## --- a/.Rproj.user/665D533C/sources/prop/E958F836
## -## +++ b/.Rproj.user/665D533C/sources/prop/E958F836
## -## @@ -1,5 +1,5 @@
## -##  {
## -## -    "cursorPosition" : "10,65",
## -## -    "scrollLine" : "0",
## -## +    "cursorPosition" : "16,18",
## -## +    "scrollLine" : "4",
## -##      "tempName" : "Untitled1"
## -##  }
## -## \ No newline at end of file
##  ## diff --git a/SessionPlan.Rmd b/SessionPlan.Rmd
## -## index 3b25a31..91a1fab 100644
## +## index 91a1fab..96e3edc 100644
##  ## --- a/SessionPlan.Rmd
##  ## +++ b/SessionPlan.Rmd
## -## @@ -3,17 +3,19 @@ title: "Intro to Git and GitHub"
## +## @@ -3,8 +3,8 @@ title: "Intro to Git and GitHub"
##  ##  author: "Tara Henechowicz"
##  ##  date: "April 25, 2022"
##  ##  output:
## -## -  beamer_presentation: default
## -##    pdf_document: default
## -## +  beamer_presentation: default
## +## -  pdf_document: default
## +##    beamer_presentation: default
## +## +  pdf_document: default
##  ##  ---
## -## -
## -## +```{r setup, include=FALSE}
## -## +knitr::opts_chunk$set(echo = FALSE)
## -## +```
## -##  #Session Outline
## -##  1. Download and Install Git help, creating github account help (first 15 minutes)
## -## -
## -##  2. Introductions (15 minutes)
## -##  3. Git Slides (see slides from U of T coders) (20 minutes)
## -##  4. Working with Git locally: edit a text file and git it
## -## -5. GitHub for Collaboration
## -## +5. Pushing to GitHub
## -## +6. GitHub features for Collaboration
## -##  
## -##    
## -##  #Step 1. Configure git
## -## @@ -80,7 +82,7 @@ you can also use `git add -A` to add all the files in your repository (remember
## -##  Let's save our first set of changes using a commit. You need to put a message with a commit.
## -##  ```{bash}
## -##  #git commit to save file to the history with a message (-m)
## -## -git commit -m "Commit 1"
## -## +git commit -m "Commit for the first time"
## -##  ```
## -##  
## -##  #Making our second commit
## -## @@ -113,82 +115,112 @@ git log
## -##  ```
## -##  
## -##  
## -## -#PART II GitHub
## -## +#Pushing repos from your computer to GitHub
## -##  
## -## -1.Create a repository on github that matches the name of your repository on your computer
## -## +Work flow: 
## -## +1. Create repo on GitHub
## -## +2. Set up connection between our computer and GitHub
## -## +3. "Push" repo from computer to GitHub
## -##  
## -## -- take our repository from our computer and "push it" to github so it can be store on their server in addition to or instead of our computer
## -## -- Why would we want to do this? 
## -## +#Create a repository on GitHub
## -## +
## -## +Create a repository on github that matches the name of your repository on your computer
## -## +
## -## +Why would we want to do this? 
## -##    1. storage (privately or publicly)
## -##    2. Public sharing of data (open science)
## -##    3. now that its on github your team members or even general public can collaborate with you on your work and you can log the changes of multiple users!
## -## +  
## -## +#Set up the connection between our computer and github
## -##  
## -## -To push our repository to github, we need to first set up the connection between the repository on our computer with the repository on github
## -## +-To push our repository to github, we need to first set up the connection between the repository on our computer with the repository on github
## -## +- go to the repository that we just made and there is the green button called "code" and copy the code from the https
## -##  
## -## -```{bash}
## -## -#go to the repository that we just made and there is the green button called "code" and copy the code from the https
## -## -  #git remote add origin https://github.com/tarahenechowicz/GitPractice.git
## -## +`git remote add origin <link for github repo>`
## -##  
## -## -#will prompt to ask for your username
## -## +- Now the computer will prompt to ask for your github username and password
## -## +- GitHub no longer takes your account password, you need to set up an authentication token
## -##  
## -## -```
## -## +#Set up the Git authentication 
## -##  
## -##  Git authetication token start up : https://docs.github.com/en/authentication/keeping-your-account-and-data-secure/creating-a-personal-access-token
## -##  
## -## -
## -## --PAT lets git control your github account
## -## -- don't lose it its like a password. 
## -## -- if you forget it you have to do this process again.
## -## -- then 
## -## -
## -## - - verify email address
## -## -    - go to settings
## -## +- verify email address
## -## +- go to settings
## -##      - developer settings, personal access tokens
## -##      - "generate a personal access token"
## -##      - recommended you change expiration, depends on the security of your data/code. 
## -## -```{bash}
## -## -#git push --set-upstream origin master #or main
## -## -#enter your github username
## -## -#copy and paste the token
## -## -```
## -## - 
## -##  
## -## -try making more changes
## -## -go through the full workflow
## -## -and do git push
## -## +- Personal access token lets git control your github account
## -## +- Don't lose it its like a password. 
## -## +- If you forget it you have to do this process again.
## -##  
## -## +#"Push" the repo from your computer to GitHub
## -##  
## -## +In your terminal use this code: 
## -## +`git push --set-upstream origin <name of branch>`
## -## +<name of branch> would be main or master depending on how your git is configured
## -##  
## -## -##Part 3 Github features
## -## -- useful for working together as a team
## -## +- next enter your github username
## -## +- paste your access token when it asks for a password
## -##  
## -## -1. Make an issue in git hub to create your README.md (15 minutes)
## -## -```{r}
## -## -#pull requests on your own account 
## -## -#pull request 
## -## -  #when there is a divergence or a branch, it takes all the divergence and squish it back into the original
## -##  
## -## -#issue
## -## - #say you found a type
## -## -  # communicate with yourself, authors and teams 
## -## -#example
## -## -#other people can make comments
## -## -#assign it to people
## -## -#just an example of the many wyas github is helpful for collaborating with scientists.
## -## -```
## -## +#GitHub features for Collaboration
## -##  
## -## -###cloning
## -## -```{bash}
## -## -# move back to your documents folder
## -## -  #review, how do you find out which folder you are currently in? 
## -## -  #how do we move to the documents folder? 
## -## +#Issues
## -## +- Issues are like little emails that you can send to yourself on GitHub or tag and assign others
## -## +- You can leave notes to let people know about errors in datasets or bugs in code. 
## -## +- You can make tasks or to-do lists. 
## -## +
## -## +#Issues practice 
## -## +Let's try to make an issue called "Set up github profile" on GitHub with the following to-do list: 
## -## + - Make a repository that is named the same as our github username
## -## + - Create a README.md file
## -## + - Inside the README.md file put in your name, bio, contact information
## -## +
## -## +#Cloning
## -## +For this lesson we are going to clone your profile repo onto your device. 
## -## +
## -## +1. Move back to your documents folder 
## -## +  - review: how do you find out which folder you are currently in? 
## -## +  - how do we move to the documents folder? 
## -## +2. On GitHub, go to the repo that you want to clone and hit the code button
## -## +3. Select https and copy that link to: 
## -## +`git clone <repo https>
## -## +
## -## +#Branches
## -## +- Allow you to split off onto a new version of work without affecting the main branch 
## -## +- Helpful for version control and collaboration if you want multiple versions of the same work
## -## +- This is an introduction to a complicated git topic - the more branches, the trickier it gets. 
## -## +- Use with caution with starting, better to focus on just using the main branch
## -## +
## -## +![](https://www.nobledesktop.com/image/gitresources/git-branches-merge.png)
## -## +
## -## +#Make a new branch
## -## +
## -## +1. Use `git chekcout` to split between branches:
## -## +
## -## +`git checkout -b <new branch name>`
## -## +
## -## +`-b` option creates a new branch, you don't need it just to switch
## -## +
## -## +- If you already have a branch created you can use `git checkout <branch name>
## -## +
## -## +- `git branch` is code to check branches and which branch you are working on 
## -## +
## -## +#Make changes on the new branch and git commit
## -## +
## -## +2. Now open one of the files in the repo and make changes in the file. 
## -## +
## -## +3. Follow the workflow to make a commit on the branch: 
## -## +  - review what is the workflow for making a commit? 
## -## +
## -## +#Push the branch to GitHub:
## -## +
## -## +4. Once you have made a commit, you can push that branch to GitHub:
## -## +`git push --set-upstream origin <branch name>
## -##  
## -## -#on github go to the repo you want to clone and hit the code button
## -## -#select https and copy that link to: 
## -## -#git clone <your readme https>
## -## -```
## -##  
## -## -####branch and pull request
## -##  ```{bash}
## -##  #make a new branch and make some change in the branch
## -##  #allows you to split off onto a new version of work without affecting the main branch. Helpful for version control if you want multiple versions of the same work. 
## -## @@ -207,20 +239,30 @@ and do git push
## -##  #now go to github and merge the changes with the main
## -##  ```
## -##  
## -## -### Forking
## -## -- straight forward
## -## -https://github.com/tarahenechowicz/Git-Demo
## -## -- make a new one
## -## +#Pull Request
## -## +
## -## +- Demo how to use Pull Request to merge changes from branch and the main
## -## +- get's more complicated with more complicated changes, code, or with multiple branches/users
## -## +
## -## +#Forking
## -## +
## -## +- go to the repo that you want to form, press "fork" button in top corner
## -## +- try forking one of my repos from: https://github.com/tarahenechowicz?tab=repositories
## -## +
## -## +#Collaboration with Github Further Reading:
## -## +
## -## +Social aspects of GitHub: 
## -## +- starring
## -## +- following
## -##  
## -## -Collaboration with Github Further Reading
## -## +- Further reading for how to collaborate on GitHub: 
## -##  https://github.com/tarahenechowicz/studyGroup/blob/gh-pages/lessons/git/collaboration/lesson.md
## -##  
## -## -# Part 4 How to use git with Rstudio
## -## +- Intermediate and Advanced GitHub topics:
## -## +https://github.com/tarahenechowicz/studyGroup/blob/gh-pages/lessons/git/
## -##  
## -## +#Demo git with Rstudio
## -##  
## -## -#Social aspect of Github
## -## -starring
## -## -following
## -##  
## -##  
## -##  
## +##  ```{r setup, include=FALSE}
## +##  knitr::opts_chunk$set(echo = FALSE)
##  ## diff --git a/SessionPlan.pdf b/SessionPlan.pdf
## -## index ba7912b..cea10e5 100644
## -## Binary files a/SessionPlan.pdf and b/SessionPlan.pdf differ
## +## deleted file mode 100644
## +## index cea10e5..0000000
## +## Binary files a/SessionPlan.pdf and /dev/null differ
##  ## On branch master
##  ## Changes to be committed:
##  ##   (use "git restore --staged <file>..." to unstage)
## -##  modified:   .Rproj.user/665D533C/sources/prop/E958F836
##  ##  modified:   SessionPlan.Rmd
## -##  modified:   SessionPlan.pdf
## +##  new file:   SessionPlan.log
## +##  deleted:    SessionPlan.pdf
## +##  new file:   SessionPlan.tex
##  \end{verbatim}
##  
## -\section{Make the commit}\label{make-the-commit}
## +\end{frame}
## +
## +\begin{frame}[fragile]{Make the commit}
##  
##  \begin{verbatim}
## -## [master 9998ddf] modified my affiliation
## -##  3 files changed, 110 insertions(+), 68 deletions(-)
## -##  rewrite SessionPlan.pdf (89%)
## +## [master ea491b3] modified my affiliation
## +##  4 files changed, 1468 insertions(+), 1 deletion(-)
## +##  create mode 100644 SessionPlan.log
## +##  delete mode 100644 SessionPlan.pdf
## +##  create mode 100644 SessionPlan.tex
##  \end{verbatim}
##  
## -\section{Let's check our log of
## -changes}\label{lets-check-our-log-of-changes}
## +\end{frame}
## +
## +\begin{frame}[fragile]{Let's check our log of changes}
##  
##  \begin{verbatim}
## +## commit ea491b36f863a8e9ad11bd07da10f4f3ac6c09ee
## +## Author: tarahenechowicz <tara.henechowicz@mail.utoronto.ca>
## +## Date:   Mon Apr 25 10:47:02 2022 -0400
## +## 
## +##     modified my affiliation
## +## 
## +## commit 70df2d0001382c267f7427a7e90fa8678e9638d8
## +## Author: tarahenechowicz <tara.henechowicz@mail.utoronto.ca>
## +## Date:   Mon Apr 25 10:47:02 2022 -0400
## +## 
## +##     Commit for the first time
## +## 
##  ## commit 9998ddf184467dbad7f1e25bd524ec5d286b4c2f
##  ## Author: tarahenechowicz <tara.henechowicz@mail.utoronto.ca>
##  ## Date:   Mon Apr 25 10:46:06 2022 -0400
## @@ -535,14 +339,16 @@ changes}\label{lets-check-our-log-of-changes}
##  ##     Initial commit
##  \end{verbatim}
##  
## -\section{Pushing repos from your computer to
## -GitHub}\label{pushing-repos-from-your-computer-to-github}
## +\end{frame}
## +
## +\begin{frame}{Pushing repos from your computer to GitHub}
##  
##  Work flow: 1. Create repo on GitHub 2. Set up connection between our
##  computer and GitHub 3. ``Push'' repo from computer to GitHub
##  
## -\section{Create a repository on
## -GitHub}\label{create-a-repository-on-github}
## +\end{frame}
## +
## +\begin{frame}{Create a repository on GitHub}
##  
##  Create a repository on github that matches the name of your repository
##  on your computer
## @@ -552,8 +358,10 @@ Public sharing of data (open science) 3. now that its on github your
##  team members or even general public can collaborate with you on your
##  work and you can log the changes of multiple users!
##  
## -\section{Set up the connection between our computer and
## -github}\label{set-up-the-connection-between-our-computer-and-github}
## +\end{frame}
## +
## +\begin{frame}[fragile]{Set up the connection between our computer and
## +github}
##  
##  -To push our repository to github, we need to first set up the
##  connection between the repository on our computer with the repository on
## @@ -572,8 +380,9 @@ button called ``code'' and copy the code from the https
##    authentication token
##  \end{itemize}
##  
## -\section{Set up the Git
## -authentication}\label{set-up-the-git-authentication}
## +\end{frame}
## +
## +\begin{frame}{Set up the Git authentication}
##  
##  Git authetication token start up :
##  \url{https://docs.github.com/en/authentication/keeping-your-account-and-data-secure/creating-a-personal-access-token}
## @@ -603,8 +412,9 @@ Git authetication token start up :
##    If you forget it you have to do this process again.
##  \end{itemize}
##  
## -\section{\texorpdfstring{``Push'' the repo from your computer to
## -GitHub}{Push the repo from your computer to GitHub}}\label{push-the-repo-from-your-computer-to-github}
## +\end{frame}
## +
## +\begin{frame}[fragile]{``Push'' the repo from your computer to GitHub}
##  
##  In your terminal use this code:
##  \texttt{git\ push\ -\/-set-upstream\ origin\ \textless{}name\ of\ branch\textgreater{}}
## @@ -618,10 +428,13 @@ would be main or master depending on how your git is configured
##    paste your access token when it asks for a password
##  \end{itemize}
##  
## -\section{GitHub features for
## -Collaboration}\label{github-features-for-collaboration}
## +\end{frame}
##  
## -\section{Issues}\label{issues}
## +\begin{frame}{GitHub features for Collaboration}
## +
## +\end{frame}
## +
## +\begin{frame}{Issues}
##  
##  \begin{itemize}
##  \tightlist
## @@ -635,14 +448,18 @@ Collaboration}\label{github-features-for-collaboration}
##    You can make tasks or to-do lists.
##  \end{itemize}
##  
## -\section{Issues practice}\label{issues-practice}
## +\end{frame}
## +
## +\begin{frame}{Issues practice}
##  
##  Let's try to make an issue called ``Set up github profile'' on GitHub
##  with the following to-do list: - Make a repository that is named the
##  same as our github username - Create a README.md file - Inside the
##  README.md file put in your name, bio, contact information
##  
## -\section{Cloning}\label{cloning}
## +\end{frame}
## +
## +\begin{frame}{Cloning}
##  
##  For this lesson we are going to clone your profile repo onto your
##  device.
## @@ -673,7 +490,9 @@ device.
##    Select https and copy that link to: `git clone 
##  \end{enumerate}
##  
## -\section{Branches}\label{branches}
## +\end{frame}
## +
## +\begin{frame}{Branches}
##  
##  \begin{itemize}
##  \tightlist
## @@ -697,7 +516,9 @@ device.
##  \caption{}
##  \end{figure}
##  
## -\section{Make a new branch}\label{make-a-new-branch}
## +\end{frame}
## +
## +\begin{frame}[fragile]{Make a new branch}
##  
##  \begin{enumerate}
##  \def\labelenumi{\arabic{enumi}.}
## @@ -719,8 +540,9 @@ switch
##    are working on
##  \end{itemize}
##  
## -\section{Make changes on the new branch and git
## -commit}\label{make-changes-on-the-new-branch-and-git-commit}
## +\end{frame}
## +
## +\begin{frame}{Make changes on the new branch and git commit}
##  
##  \begin{enumerate}
##  \def\labelenumi{\arabic{enumi}.}
## @@ -737,7 +559,9 @@ commit}\label{make-changes-on-the-new-branch-and-git-commit}
##    review what is the workflow for making a commit?
##  \end{itemize}
##  
## -\section{Push the branch to GitHub:}\label{push-the-branch-to-github}
## +\end{frame}
## +
## +\begin{frame}{Push the branch to GitHub:}
##  
##  \begin{enumerate}
##  \def\labelenumi{\arabic{enumi}.}
## @@ -748,7 +572,9 @@ commit}\label{make-changes-on-the-new-branch-and-git-commit}
##    push --set-upstream origin 
##  \end{enumerate}
##  
## -\section{Pull Request}\label{pull-request}
## +\end{frame}
## +
## +\begin{frame}{Pull Request}
##  
##  \begin{itemize}
##  \tightlist
## @@ -759,7 +585,9 @@ commit}\label{make-changes-on-the-new-branch-and-git-commit}
##    multiple branches/users
##  \end{itemize}
##  
## -\section{Forking}\label{forking}
## +\end{frame}
## +
## +\begin{frame}{Forking}
##  
##  \begin{itemize}
##  \tightlist
## @@ -771,8 +599,9 @@ commit}\label{make-changes-on-the-new-branch-and-git-commit}
##    \url{https://github.com/tarahenechowicz?tab=repositories}
##  \end{itemize}
##  
## -\section{Collaboration with Github Further
## -Reading:}\label{collaboration-with-github-further-reading}
## +\end{frame}
## +
## +\begin{frame}{Collaboration with Github Further Reading:}
##  
##  Social aspects of GitHub: - starring - following
##  
## @@ -785,6 +614,10 @@ Social aspects of GitHub: - starring - following
##    \url{https://github.com/tarahenechowicz/studyGroup/blob/gh-pages/lessons/git/}
##  \end{itemize}
##  
## -\section{Demo git with Rstudio}\label{demo-git-with-rstudio}
## +\end{frame}
## +
## +\begin{frame}{Demo git with Rstudio}
## +
## +\end{frame}
##  
##  \end{document}
## On branch master
## Changes to be committed:
##   (use "git restore --staged <file>..." to unstage)
##  modified:   SessionPlan.Rmd
##  modified:   SessionPlan.log
##  modified:   SessionPlan.tex
\end{verbatim}

\end{frame}

\begin{frame}[fragile]{Make the commit}

\begin{verbatim}
## [master 9376833] modified my affiliation
##  3 files changed, 1133 insertions(+), 770 deletions(-)
\end{verbatim}

\end{frame}

\begin{frame}[fragile]{Let's check our log of changes}

\begin{verbatim}
## commit 93768336b8da3734972e5196e1379a7448729e7f
## Author: tarahenechowicz <tara.henechowicz@mail.utoronto.ca>
## Date:   Mon Apr 25 10:48:17 2022 -0400
## 
##     modified my affiliation
## 
## commit f6fdefad8cc4fbed49c30b5fb7ebd0921d15bdb0
## Author: tarahenechowicz <tara.henechowicz@mail.utoronto.ca>
## Date:   Mon Apr 25 10:48:17 2022 -0400
## 
##     Commit for the first time
## 
## commit ea491b36f863a8e9ad11bd07da10f4f3ac6c09ee
## Author: tarahenechowicz <tara.henechowicz@mail.utoronto.ca>
## Date:   Mon Apr 25 10:47:02 2022 -0400
## 
##     modified my affiliation
## 
## commit 70df2d0001382c267f7427a7e90fa8678e9638d8
## Author: tarahenechowicz <tara.henechowicz@mail.utoronto.ca>
## Date:   Mon Apr 25 10:47:02 2022 -0400
## 
##     Commit for the first time
## 
## commit 9998ddf184467dbad7f1e25bd524ec5d286b4c2f
## Author: tarahenechowicz <tara.henechowicz@mail.utoronto.ca>
## Date:   Mon Apr 25 10:46:06 2022 -0400
## 
##     modified my affiliation
## 
## commit adc6a3d934fa493ddfea840f1aa7d78a102b2c3c
## Author: tarahenechowicz <tara.henechowicz@mail.utoronto.ca>
## Date:   Mon Apr 25 10:46:06 2022 -0400
## 
##     Commit for the first time
## 
## commit 11fd6568a4766c90d44d6cfd8085fd2912767e03
## Author: tarahenechowicz <tara.henechowicz@mail.utoronto.ca>
## Date:   Mon Apr 25 09:36:21 2022 -0400
## 
##     modified my affiliation
## 
## commit d827503f54eaa0efba7ceb297d0488a26760602e
## Author: tarahenechowicz <tara.henechowicz@mail.utoronto.ca>
## Date:   Mon Apr 25 09:36:21 2022 -0400
## 
##     Commit 1
## 
## commit 0be27699d0be6974d6173dbc046729917b37d2c3
## Author: tarahenechowicz <tara.henechowicz@mail.utoronto.ca>
## Date:   Mon Apr 25 09:21:06 2022 -0400
## 
##     Initial commit
## 
## commit a0a95ab0935b0df63dab068b39c5fb69919a6b73
## Author: tarahenechowicz <tara.henechowicz@mail.utoronto.ca>
## Date:   Mon Apr 25 09:20:19 2022 -0400
## 
##     Initial commit
## 
## commit f05a78beacc177cb74a8fd7f1720b1f4219f008d
## Author: tarahenechowicz <tara.henechowicz@mail.utoronto.ca>
## Date:   Mon Apr 25 09:19:49 2022 -0400
## 
##     Initial commit
## 
## commit 044d6ad769a04c9bd6f60f49155937266d1e8d5b
## Author: Tara Henechowicz <tara.henechowicz@mail.utoronto.ca>
## Date:   Thu Apr 21 19:52:55 2022 -0400
## 
##     Created a session plan
## 
## commit c75bdd1e338193991ba701a7406c3f4863dbd8e1
## Author: Tara Henechowicz <tara.henechowicz@mail.utoronto.ca>
## Date:   Thu Apr 21 14:13:20 2022 -0400
## 
##     added workshop instructions
## 
## commit c3843a2588678bee6120dbc29f145ac8641bccc2
## Author: Tara Henechowicz <tara.henechowicz@mail.utoronto.ca>
## Date:   Thu Apr 21 13:29:58 2022 -0400
## 
##     Added my bio
## 
## commit be076c9021b5306bf0ddad550d144e7a9c8556f0
## Author: Tara Henechowicz <tara.henechowicz@mail.utoronto.ca>
## Date:   Thu Apr 21 13:28:34 2022 -0400
## 
##     Initial commit
\end{verbatim}

\end{frame}

\begin{frame}{Pushing repos from your computer to GitHub}

Work flow: 1. Create repo on GitHub 2. Set up connection between our
computer and GitHub 3. ``Push'' repo from computer to GitHub

\end{frame}

\begin{frame}{Create a repository on GitHub}

Create a repository on github that matches the name of your repository
on your computer

Why would we want to do this? 1. storage (privately or publicly) 2.
Public sharing of data (open science) 3. now that its on github your
team members or even general public can collaborate with you on your
work and you can log the changes of multiple users!

\end{frame}

\begin{frame}[fragile]{Set up the connection between our computer and
github}

-To push our repository to github, we need to first set up the
connection between the repository on our computer with the repository on
github - go to the repository that we just made and there is the green
button called ``code'' and copy the code from the https

\texttt{git\ remote\ add\ origin\ \textless{}link\ for\ github\ repo\textgreater{}}

\begin{itemize}
\tightlist
\item
  Now the computer will prompt to ask for your github username and
  password
\item
  GitHub no longer takes your account password, you need to set up an
  authentication token
\end{itemize}

\end{frame}

\begin{frame}{Set up the Git authentication}

Git authetication token start up :
\url{https://docs.github.com/en/authentication/keeping-your-account-and-data-secure/creating-a-personal-access-token}

\begin{itemize}
\tightlist
\item
  verify email address
\item
  go to settings

  \begin{itemize}
  \tightlist
  \item
    developer settings, personal access tokens
  \item
    ``generate a personal access token''
  \item
    recommended you change expiration, depends on the security of your
    data/code.
  \end{itemize}
\item
  Personal access token lets git control your github account
\item
  Don't lose it its like a password.
\item
  If you forget it you have to do this process again.
\end{itemize}

\end{frame}

\begin{frame}[fragile]{``Push'' the repo from your computer to GitHub}

In your terminal use this code:
\texttt{git\ push\ -\/-set-upstream\ origin\ \textless{}name\ of\ branch\textgreater{}}
would be main or master depending on how your git is configured

\begin{itemize}
\tightlist
\item
  next enter your github username
\item
  paste your access token when it asks for a password
\end{itemize}

\end{frame}

\begin{frame}{GitHub features for Collaboration}

\end{frame}

\begin{frame}{Issues}

\begin{itemize}
\tightlist
\item
  Issues are like little emails that you can send to yourself on GitHub
  or tag and assign others
\item
  You can leave notes to let people know about errors in datasets or
  bugs in code.
\item
  You can make tasks or to-do lists.
\end{itemize}

\end{frame}

\begin{frame}{Issues practice}

Let's try to make an issue called ``Set up github profile'' on GitHub
with the following to-do list: - Make a repository that is named the
same as our github username - Create a README.md file - Inside the
README.md file put in your name, bio, contact information

\end{frame}

\begin{frame}{Cloning}

For this lesson we are going to clone your profile repo onto your
device.

\begin{enumerate}
\def\labelenumi{\arabic{enumi}.}
\tightlist
\item
  Move back to your documents folder
\end{enumerate}

\begin{itemize}
\tightlist
\item
  review: how do you find out which folder you are currently in?
\item
  how do we move to the documents folder?
\end{itemize}

\begin{enumerate}
\def\labelenumi{\arabic{enumi}.}
\setcounter{enumi}{1}
\tightlist
\item
  On GitHub, go to the repo that you want to clone and hit the code
  button
\item
  Select https and copy that link to: `git clone 
\end{enumerate}

\end{frame}

\begin{frame}{Branches}

\begin{itemize}
\tightlist
\item
  Allow you to split off onto a new version of work without affecting
  the main branch
\item
  Helpful for version control and collaboration if you want multiple
  versions of the same work
\item
  This is an introduction to a complicated git topic - the more
  branches, the trickier it gets.
\item
  Use with caution with starting, better to focus on just using the main
  branch
\end{itemize}

\end{frame}

\begin{frame}[fragile]{Make a new branch}

\begin{enumerate}
\def\labelenumi{\arabic{enumi}.}
\tightlist
\item
  Use \texttt{git\ chekcout} to split between branches:
\end{enumerate}

\texttt{git\ checkout\ -b\ \textless{}new\ branch\ name\textgreater{}}

\texttt{-b} option creates a new branch, you don't need it just to
switch

\begin{itemize}
\item
  If you already have a branch created you can use `git checkout 
\item
  \texttt{git\ branch} is code to check branches and which branch you
  are working on
\end{itemize}

\end{frame}

\begin{frame}{Make changes on the new branch and git commit}

\begin{enumerate}
\def\labelenumi{\arabic{enumi}.}
\setcounter{enumi}{1}
\item
  Now open one of the files in the repo and make changes in the file.
\item
  Follow the workflow to make a commit on the branch:
\end{enumerate}

\begin{itemize}
\tightlist
\item
  review what is the workflow for making a commit?
\end{itemize}

\end{frame}

\begin{frame}{Push the branch to GitHub:}

\begin{enumerate}
\def\labelenumi{\arabic{enumi}.}
\setcounter{enumi}{3}
\tightlist
\item
  Once you have made a commit, you can push that branch to GitHub: `git
  push --set-upstream origin 
\end{enumerate}

\end{frame}

\begin{frame}{Pull Request}

\begin{itemize}
\tightlist
\item
  Demo how to use Pull Request to merge changes from branch and the main
\item
  get's more complicated with more complicated changes, code, or with
  multiple branches/users
\end{itemize}

\end{frame}

\begin{frame}{Forking}

\begin{itemize}
\tightlist
\item
  go to the repo that you want to form, press ``fork'' button in top
  corner
\item
  try forking one of my repos from:
  \url{https://github.com/tarahenechowicz?tab=repositories}
\end{itemize}

\end{frame}

\begin{frame}{Collaboration with Github Further Reading:}

Social aspects of GitHub: - starring - following

\begin{itemize}
\item
  Further reading for how to collaborate on GitHub:
  \url{https://github.com/tarahenechowicz/studyGroup/blob/gh-pages/lessons/git/collaboration/lesson.md}
\item
  Intermediate and Advanced GitHub topics:
  \url{https://github.com/tarahenechowicz/studyGroup/blob/gh-pages/lessons/git/}
\end{itemize}

\end{frame}

\begin{frame}{Demo git with Rstudio}

\end{frame}

\end{document}
