\documentclass[ignorenonframetext,]{beamer}
\setbeamertemplate{caption}[numbered]
\setbeamertemplate{caption label separator}{: }
\setbeamercolor{caption name}{fg=normal text.fg}
\beamertemplatenavigationsymbolsempty
\usepackage{lmodern}
\usepackage{amssymb,amsmath}
\usepackage{ifxetex,ifluatex}
\usepackage{fixltx2e} % provides \textsubscript
\ifnum 0\ifxetex 1\fi\ifluatex 1\fi=0 % if pdftex
  \usepackage[T1]{fontenc}
  \usepackage[utf8]{inputenc}
\else % if luatex or xelatex
  \ifxetex
    \usepackage{mathspec}
  \else
    \usepackage{fontspec}
  \fi
  \defaultfontfeatures{Ligatures=TeX,Scale=MatchLowercase}
\fi
% use upquote if available, for straight quotes in verbatim environments
\IfFileExists{upquote.sty}{\usepackage{upquote}}{}
% use microtype if available
\IfFileExists{microtype.sty}{%
\usepackage{microtype}
\UseMicrotypeSet[protrusion]{basicmath} % disable protrusion for tt fonts
}{}
\newif\ifbibliography
\hypersetup{
            pdftitle={Intro to Git and GitHub},
            pdfauthor={Tara Henechowicz},
            pdfborder={0 0 0},
            breaklinks=true}
\urlstyle{same}  % don't use monospace font for urls
\usepackage{graphicx,grffile}
\makeatletter
\def\maxwidth{\ifdim\Gin@nat@width>\linewidth\linewidth\else\Gin@nat@width\fi}
\def\maxheight{\ifdim\Gin@nat@height>\textheight0.8\textheight\else\Gin@nat@height\fi}
\makeatother
% Scale images if necessary, so that they will not overflow the page
% margins by default, and it is still possible to overwrite the defaults
% using explicit options in \includegraphics[width, height, ...]{}
\setkeys{Gin}{width=\maxwidth,height=\maxheight,keepaspectratio}

% Prevent slide breaks in the middle of a paragraph:
\widowpenalties 1 10000
\raggedbottom

\AtBeginPart{
  \let\insertpartnumber\relax
  \let\partname\relax
  \frame{\partpage}
}
\AtBeginSection{
  \ifbibliography
  \else
    \let\insertsectionnumber\relax
    \let\sectionname\relax
    \frame{\sectionpage}
  \fi
}
\AtBeginSubsection{
  \let\insertsubsectionnumber\relax
  \let\subsectionname\relax
  \frame{\subsectionpage}
}

\setlength{\parindent}{0pt}
\setlength{\parskip}{6pt plus 2pt minus 1pt}
\setlength{\emergencystretch}{3em}  % prevent overfull lines
\providecommand{\tightlist}{%
  \setlength{\itemsep}{0pt}\setlength{\parskip}{0pt}}
\setcounter{secnumdepth}{0}

\title{Intro to Git and GitHub}
\author{Tara Henechowicz}
\date{April 25, 2022}

\begin{document}
\frame{\titlepage}

\begin{frame}{Session Outline}

\begin{enumerate}
\def\labelenumi{\arabic{enumi}.}
\tightlist
\item
  Download and Install Git help, creating github account help (first 15
  minutes)
\item
  Introductions (15 minutes)
\item
  Git Slides (see slides from U of T coders) (20 minutes)
\item
  Working with Git locally: edit a text file and git it
\item
  Pushing to GitHub
\item
  GitHub features for Collaboration
\end{enumerate}

\end{frame}

\begin{frame}{Step 1. Configure git}

\end{frame}

\begin{frame}[fragile]{Step 2. Create a Repository on your computer with
gitGit}

\begin{verbatim}
## /Users/tarahenechowicz/Documents/RTeaching/IntroGitGitHub
## bash: line 7: cd: Documents: No such file or directory
## mkdir: GitPractice2: File exists
\end{verbatim}

\end{frame}

\begin{frame}[fragile]{Step 3. Git Init (Initialize the Repo)}

\begin{verbatim}
## Reinitialized existing Git repository in /Users/tarahenechowicz/Documents/RTeaching/IntroGitGitHub/.git/
## master
\end{verbatim}

\end{frame}

\begin{frame}{Adding files to the Repo}

\begin{itemize}
\tightlist
\item
  Let's create a text file to store some data to work on
\end{itemize}

\begin{enumerate}
\def\labelenumi{\arabic{enumi}.}
\item
  Create a text file named mydata.txt
\item
  Open the file and add:
\end{enumerate}

\begin{itemize}
\tightlist
\item
  your name
\item
  your program
\item
  your research area
\end{itemize}

\begin{enumerate}
\def\labelenumi{\arabic{enumi}.}
\setcounter{enumi}{2}
\tightlist
\item
  Save the file.
\end{enumerate}

\end{frame}

\begin{frame}[fragile]{Making our first commit}

\begin{verbatim}
## On branch master
## Changes not staged for commit:
##   (use "git add <file>..." to update what will be committed)
##   (use "git restore <file>..." to discard changes in working directory)
##  modified:   mydata.txt
## 
## no changes added to commit (use "git add" and/or "git commit -a")
\end{verbatim}

you can also use \texttt{git\ add\ -A} to add all the files in your
repository (remember we only have one right now)

\end{frame}

\begin{frame}[fragile]{Making our first commit}

Let's save our first set of changes using a commit. You need to put a
message with a commit.

\begin{verbatim}
## [master 70df2d0] Commit for the first time
##  1 file changed, 1 insertion(+)
\end{verbatim}

\end{frame}

\begin{frame}{Making our second commit}

Now we are going to back to the text file and make a change. - for
example, I'm going to add my affiliation in the first line - you can add
your programming languages and experiences - or add a sentence about why
you want to use git

\end{frame}

\begin{frame}[fragile]{Check activity and changes with git status or git
diff}

\begin{verbatim}
## On branch master
## Changes not staged for commit:
##   (use "git add/rm <file>..." to update what will be committed)
##   (use "git restore <file>..." to discard changes in working directory)
##  modified:   SessionPlan.Rmd
##  deleted:    SessionPlan.pdf
## 
## Untracked files:
##   (use "git add <file>..." to include in what will be committed)
##  SessionPlan.log
##  SessionPlan.tex
## 
## no changes added to commit (use "git add" and/or "git commit -a")
## diff --git a/SessionPlan.Rmd b/SessionPlan.Rmd
## index 91a1fab..96e3edc 100644
## --- a/SessionPlan.Rmd
## +++ b/SessionPlan.Rmd
## @@ -3,8 +3,8 @@ title: "Intro to Git and GitHub"
##  author: "Tara Henechowicz"
##  date: "April 25, 2022"
##  output:
## -  pdf_document: default
##    beamer_presentation: default
## +  pdf_document: default
##  ---
##  ```{r setup, include=FALSE}
##  knitr::opts_chunk$set(echo = FALSE)
## diff --git a/SessionPlan.pdf b/SessionPlan.pdf
## deleted file mode 100644
## index cea10e5..0000000
## Binary files a/SessionPlan.pdf and /dev/null differ
## On branch master
## Changes to be committed:
##   (use "git restore --staged <file>..." to unstage)
##  modified:   SessionPlan.Rmd
##  new file:   SessionPlan.log
##  deleted:    SessionPlan.pdf
##  new file:   SessionPlan.tex
\end{verbatim}

\end{frame}

\begin{frame}[fragile]{Make the commit}

\begin{verbatim}
## [master ea491b3] modified my affiliation
##  4 files changed, 1468 insertions(+), 1 deletion(-)
##  create mode 100644 SessionPlan.log
##  delete mode 100644 SessionPlan.pdf
##  create mode 100644 SessionPlan.tex
\end{verbatim}

\end{frame}

\begin{frame}[fragile]{Let's check our log of changes}

\begin{verbatim}
## commit ea491b36f863a8e9ad11bd07da10f4f3ac6c09ee
## Author: tarahenechowicz <tara.henechowicz@mail.utoronto.ca>
## Date:   Mon Apr 25 10:47:02 2022 -0400
## 
##     modified my affiliation
## 
## commit 70df2d0001382c267f7427a7e90fa8678e9638d8
## Author: tarahenechowicz <tara.henechowicz@mail.utoronto.ca>
## Date:   Mon Apr 25 10:47:02 2022 -0400
## 
##     Commit for the first time
## 
## commit 9998ddf184467dbad7f1e25bd524ec5d286b4c2f
## Author: tarahenechowicz <tara.henechowicz@mail.utoronto.ca>
## Date:   Mon Apr 25 10:46:06 2022 -0400
## 
##     modified my affiliation
## 
## commit adc6a3d934fa493ddfea840f1aa7d78a102b2c3c
## Author: tarahenechowicz <tara.henechowicz@mail.utoronto.ca>
## Date:   Mon Apr 25 10:46:06 2022 -0400
## 
##     Commit for the first time
## 
## commit 11fd6568a4766c90d44d6cfd8085fd2912767e03
## Author: tarahenechowicz <tara.henechowicz@mail.utoronto.ca>
## Date:   Mon Apr 25 09:36:21 2022 -0400
## 
##     modified my affiliation
## 
## commit d827503f54eaa0efba7ceb297d0488a26760602e
## Author: tarahenechowicz <tara.henechowicz@mail.utoronto.ca>
## Date:   Mon Apr 25 09:36:21 2022 -0400
## 
##     Commit 1
## 
## commit 0be27699d0be6974d6173dbc046729917b37d2c3
## Author: tarahenechowicz <tara.henechowicz@mail.utoronto.ca>
## Date:   Mon Apr 25 09:21:06 2022 -0400
## 
##     Initial commit
## 
## commit a0a95ab0935b0df63dab068b39c5fb69919a6b73
## Author: tarahenechowicz <tara.henechowicz@mail.utoronto.ca>
## Date:   Mon Apr 25 09:20:19 2022 -0400
## 
##     Initial commit
## 
## commit f05a78beacc177cb74a8fd7f1720b1f4219f008d
## Author: tarahenechowicz <tara.henechowicz@mail.utoronto.ca>
## Date:   Mon Apr 25 09:19:49 2022 -0400
## 
##     Initial commit
## 
## commit 044d6ad769a04c9bd6f60f49155937266d1e8d5b
## Author: Tara Henechowicz <tara.henechowicz@mail.utoronto.ca>
## Date:   Thu Apr 21 19:52:55 2022 -0400
## 
##     Created a session plan
## 
## commit c75bdd1e338193991ba701a7406c3f4863dbd8e1
## Author: Tara Henechowicz <tara.henechowicz@mail.utoronto.ca>
## Date:   Thu Apr 21 14:13:20 2022 -0400
## 
##     added workshop instructions
## 
## commit c3843a2588678bee6120dbc29f145ac8641bccc2
## Author: Tara Henechowicz <tara.henechowicz@mail.utoronto.ca>
## Date:   Thu Apr 21 13:29:58 2022 -0400
## 
##     Added my bio
## 
## commit be076c9021b5306bf0ddad550d144e7a9c8556f0
## Author: Tara Henechowicz <tara.henechowicz@mail.utoronto.ca>
## Date:   Thu Apr 21 13:28:34 2022 -0400
## 
##     Initial commit
\end{verbatim}

\end{frame}

\begin{frame}{Pushing repos from your computer to GitHub}

Work flow: 1. Create repo on GitHub 2. Set up connection between our
computer and GitHub 3. ``Push'' repo from computer to GitHub

\end{frame}

\begin{frame}{Create a repository on GitHub}

Create a repository on github that matches the name of your repository
on your computer

Why would we want to do this? 1. storage (privately or publicly) 2.
Public sharing of data (open science) 3. now that its on github your
team members or even general public can collaborate with you on your
work and you can log the changes of multiple users!

\end{frame}

\begin{frame}[fragile]{Set up the connection between our computer and
github}

-To push our repository to github, we need to first set up the
connection between the repository on our computer with the repository on
github - go to the repository that we just made and there is the green
button called ``code'' and copy the code from the https

\texttt{git\ remote\ add\ origin\ \textless{}link\ for\ github\ repo\textgreater{}}

\begin{itemize}
\tightlist
\item
  Now the computer will prompt to ask for your github username and
  password
\item
  GitHub no longer takes your account password, you need to set up an
  authentication token
\end{itemize}

\end{frame}

\begin{frame}{Set up the Git authentication}

Git authetication token start up :
\url{https://docs.github.com/en/authentication/keeping-your-account-and-data-secure/creating-a-personal-access-token}

\begin{itemize}
\tightlist
\item
  verify email address
\item
  go to settings

  \begin{itemize}
  \tightlist
  \item
    developer settings, personal access tokens
  \item
    ``generate a personal access token''
  \item
    recommended you change expiration, depends on the security of your
    data/code.
  \end{itemize}
\item
  Personal access token lets git control your github account
\item
  Don't lose it its like a password.
\item
  If you forget it you have to do this process again.
\end{itemize}

\end{frame}

\begin{frame}[fragile]{``Push'' the repo from your computer to GitHub}

In your terminal use this code:
\texttt{git\ push\ -\/-set-upstream\ origin\ \textless{}name\ of\ branch\textgreater{}}
would be main or master depending on how your git is configured

\begin{itemize}
\tightlist
\item
  next enter your github username
\item
  paste your access token when it asks for a password
\end{itemize}

\end{frame}

\begin{frame}{GitHub features for Collaboration}

\end{frame}

\begin{frame}{Issues}

\begin{itemize}
\tightlist
\item
  Issues are like little emails that you can send to yourself on GitHub
  or tag and assign others
\item
  You can leave notes to let people know about errors in datasets or
  bugs in code.
\item
  You can make tasks or to-do lists.
\end{itemize}

\end{frame}

\begin{frame}{Issues practice}

Let's try to make an issue called ``Set up github profile'' on GitHub
with the following to-do list: - Make a repository that is named the
same as our github username - Create a README.md file - Inside the
README.md file put in your name, bio, contact information

\end{frame}

\begin{frame}{Cloning}

For this lesson we are going to clone your profile repo onto your
device.

\begin{enumerate}
\def\labelenumi{\arabic{enumi}.}
\tightlist
\item
  Move back to your documents folder
\end{enumerate}

\begin{itemize}
\tightlist
\item
  review: how do you find out which folder you are currently in?
\item
  how do we move to the documents folder?
\end{itemize}

\begin{enumerate}
\def\labelenumi{\arabic{enumi}.}
\setcounter{enumi}{1}
\tightlist
\item
  On GitHub, go to the repo that you want to clone and hit the code
  button
\item
  Select https and copy that link to: `git clone 
\end{enumerate}

\end{frame}

\begin{frame}{Branches}

\begin{itemize}
\tightlist
\item
  Allow you to split off onto a new version of work without affecting
  the main branch
\item
  Helpful for version control and collaboration if you want multiple
  versions of the same work
\item
  This is an introduction to a complicated git topic - the more
  branches, the trickier it gets.
\item
  Use with caution with starting, better to focus on just using the main
  branch
\end{itemize}

\begin{figure}
\centering
\includegraphics{https://www.nobledesktop.com/image/gitresources/git-branches-merge.png}
\caption{}
\end{figure}

\end{frame}

\begin{frame}[fragile]{Make a new branch}

\begin{enumerate}
\def\labelenumi{\arabic{enumi}.}
\tightlist
\item
  Use \texttt{git\ chekcout} to split between branches:
\end{enumerate}

\texttt{git\ checkout\ -b\ \textless{}new\ branch\ name\textgreater{}}

\texttt{-b} option creates a new branch, you don't need it just to
switch

\begin{itemize}
\item
  If you already have a branch created you can use `git checkout 
\item
  \texttt{git\ branch} is code to check branches and which branch you
  are working on
\end{itemize}

\end{frame}

\begin{frame}{Make changes on the new branch and git commit}

\begin{enumerate}
\def\labelenumi{\arabic{enumi}.}
\setcounter{enumi}{1}
\item
  Now open one of the files in the repo and make changes in the file.
\item
  Follow the workflow to make a commit on the branch:
\end{enumerate}

\begin{itemize}
\tightlist
\item
  review what is the workflow for making a commit?
\end{itemize}

\end{frame}

\begin{frame}{Push the branch to GitHub:}

\begin{enumerate}
\def\labelenumi{\arabic{enumi}.}
\setcounter{enumi}{3}
\tightlist
\item
  Once you have made a commit, you can push that branch to GitHub: `git
  push --set-upstream origin 
\end{enumerate}

\end{frame}

\begin{frame}{Pull Request}

\begin{itemize}
\tightlist
\item
  Demo how to use Pull Request to merge changes from branch and the main
\item
  get's more complicated with more complicated changes, code, or with
  multiple branches/users
\end{itemize}

\end{frame}

\begin{frame}{Forking}

\begin{itemize}
\tightlist
\item
  go to the repo that you want to form, press ``fork'' button in top
  corner
\item
  try forking one of my repos from:
  \url{https://github.com/tarahenechowicz?tab=repositories}
\end{itemize}

\end{frame}

\begin{frame}{Collaboration with Github Further Reading:}

Social aspects of GitHub: - starring - following

\begin{itemize}
\item
  Further reading for how to collaborate on GitHub:
  \url{https://github.com/tarahenechowicz/studyGroup/blob/gh-pages/lessons/git/collaboration/lesson.md}
\item
  Intermediate and Advanced GitHub topics:
  \url{https://github.com/tarahenechowicz/studyGroup/blob/gh-pages/lessons/git/}
\end{itemize}

\end{frame}

\begin{frame}{Demo git with Rstudio}

\end{frame}

\end{document}
